\documentclass{mathnotes}

\name{Jacky Lee}
\notetitle{Complex Analysis Notes}
\notedate{\today}

\newcommand{\pder}[2][]{\frac{\partial#1}{\partial#2}}
\newcommand{\ipder}[2][]{\partial#1/\partial#2}

\begin{document}
\begin{center}
  \vspace*{20pt}
  \LARGE{Complex Analysis Notes}
\end{center}

\begin{bdefi}
  A function $f:\Omega\to\CC$ is \define{holomorphic} at $z_0\in\Omega$ if
  $$\lim_{h\to0}\frac{f(z_0+h)-f(z_0)}{h}$$
  exists. This limit, if it exists, is denoted by $f'(z_0)$ and is called the
  \define{derivative} of $f$ at $z_0$.
\end{bdefi}

\begin{defi}
  A function $f:\Omega\to\CC$ is \define{entire} if it is holomorphic on all of
  $\CC$.
\end{defi}

\begin{defi}
  The terms \define{regular} and \define{complex differentiable} are sometimes
  also used to refer to holomorphic functions.
\end{defi}

\begin{prop}
  A function $f$ is holomorphic at $z_0\in\Omega$ if and only if there exists a
  complex number $a$ such that
  $$f(z_0+h)-f(z_0)-ah=h\psi(h)$$
  where $\psi$ is a function defined for all small $h$ and
  $\lim_{h\to0}\psi(h)=0$. Note that $a=f'(z_0)$.
\end{prop}

\begin{prop}
  Let $f$ and $g$ be holomorphic in $\Omega$.
  \begin{enumerate}
    \item $f+g$ is holomorphic in $\Omega$ and
      $$(f+g)'=f'+g'$$
    \item $fg$ is holomorphic in $\Omega$ and
      $$(fg)'=f'g+fg'$$
    \item If $g(z_0)\ne0$, then $f/g$ is holomorphic at $z_0$ and
      $$(f/g)'=(f'g+fg')/g^2$$
    \item If $f:\Omega\to U$ and $g:U\to\CC$ are holomorphic, then $g\circ f$
      is holomorphic in $\Omega$ and
      $$(g\circ f)'(z)=g'(f(z))f'(z),\ \forall z\in\Omega$$.
  \end{enumerate}
\end{prop}

\begin{ex}
  Consider the function $f(z)=\bar z$. This function is not holomorphic since
  $$\frac{f(z_0+h)-f(z_0)}{h}=\frac{\bar h}{h}$$
  which has no limit as $h\to0$. This can be seen by first setting $h$ real in
  contrast to first setting $h$ imaginary.
\end{ex}

\begin{note}
  Note $f$ corresponds to the function $F:(x,y)\mapsto(x,-y)$ in $\RR^2$. This
  function is real differentiable and its Jacobian is constant. Furthermore,
  it is infinitely differentiable. Thus, real differentiability does not imply
  complex differentiability. Also, note that in real differentiation, the
  derivative is a matrix (the Jacobian) whereas in complex differentiation, the
  derivative is a complex number.
\end{note}

\begin{bprop}[Cauchy-Riemann Equations]
  Let $f$ be a complex-valued function. Suppose we decompose $f$ into a real
  and imaginary component, i.e. $f(x,y)=u(x,y)+iv(x,y)$, and the partial
  derivatives of $u$ and $v$ exist. Then they satisfy the equations
  $$\pder[u]{x}=\pder[v]{y}\hspace{50pt}\pder[u]{y}=-\pder[v]{x}$$
\end{bprop}

\begin{defi}
  We define the following two differential operators.
  $$\pder{z}=\frac{1}{2}\left(\pder{x}+\frac{1}{i}\pder{y}\right) \hspace{50pt}
  \pder{\bar z}=\frac{1}{2}\left(\pder{x}-\frac{1}{i}\pder{y}\right)$$
\end{defi}

\begin{prop}
  Let $f$ be a function that is holomorphic at $z_0$, then
  $$\pder[f]{\bar z}(z_0)=0\hspace{50pt}
  f'(z_0)=\pder[f]{z}(z_0)=2\pder[u]{z}(z_0)$$
\end{prop}

\begin{prop}
  Let $f$ be a function that is holomorphic at $z_0=x_0+iy_0$. If we let
  $F(x,y)=f(z)$ where $z=x+iy$, then $F$ is real differentiable and
  $$\det J_F(x_0,y_0)=|f'(z_0)|^2$$
\end{prop}

\begin{thm}
  Suppose $f=u+iv$ is a complex-valued function defined on an open set
  $\Omega$. If $u$ and $v$ are continuously differentiable and satisfy the
  Cauchy-Riemann equations on $\Omega$, then $f$ is holomorphic on $\Omega$ and
  $f'(z)=\ipder[f]{z}$.
\end{thm}

\begin{thm}
  Given a power series $\sum_{n=0}^\infty a_nz^n$, there exists $0<R<\infty$
  such that
  \begin{enumerate}
    \item if $|z|<R$, the series converges absolutely
    \item if $|z|>R$, the series diverges
  \end{enumerate}
  Moreover, if we use the convention that $1/0=\infty$ and $1/\infty=0$, then
  $R$ is given by Hadamard's formula
  $$\frac{1}{R}=\lim\sup|a_n|^{1/n}$$
\end{thm}

\begin{defi}
  The number $R$ is called the \define{radius of convergence} of the power
  series, and the region $|z|<R$ the \define{disc of convergence}.
\end{defi}

\begin{note}
  On the boundary of the disc of convergence, $|z|=R$, one can have either
  convergence or divergence.
\end{note}

\begin{thm}
  The power series $f(z)=\sum_{n=0}^\infty a_nz^n$ defines a holomorphic
  function in its disc of convergence. The derivative of $f$ is also a power
  series obtained by differentiating term by term the series for $f$, that is,
  $$f'(z)=\sum_{n=0}^\infty na_nz^{n-1}$$
  Moreover, $f'$ has the same radius of convergence as $f$.
\end{thm}

\begin{cor}
  A power series is infinitely complex differentiable in its disc of
  convergence, and the higher derivatives are also power series obtained by
  termwise differentiation.
\end{cor}

\begin{note}
  A power series centered at $z_0\in\CC$ is an expression of the form
  $$f(z)=\sum_{n=0}^\infty a_n(z-z_0)^n$$
  The disc of convergence is now centered at $z_0$ and its radius is still
  given by Hadamard's formula. In fact, if
  $$g(z)=\sum_{n=0}^\infty a_nz^n$$
  then $f(z)=g(z-z_0)$.
\end{note}

\begin{defi}
  A function $f$ defined on an open set $\Omega$ is said to be
  \define{analytic} at a point $z_0\in\Omega$ if there exists a power series
  $\sum a_n(z-z_0)^n$ centered at $z_0$, with positive radius of convergence,
  such that
  $$f(z)=\sum_{n=0}^\infty a_n(z-z_0)^n$$
  for all $z$ in a neighborhood of $z_0$.
\end{defi}

\end{document}
