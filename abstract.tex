\documentclass[11pt,letterpaper,boxed]{hmcpset}
\usepackage[margin=1in,headheight=14pt]{geometry}
\usepackage{amsfonts, amsmath, amssymb, enumerate, fancyhdr, gensymb, lastpage, mathtools}
\usepackage[usenames, dvipsnames, svgnames]{xcolor}
\usepackage{amsthm, thmtools}
\usepackage[framemethod=TikZ]{mdframed}

\pagestyle{fancy}
\lhead{Jacky Lee}
\chead{Abstract Algebra Notes}
\rhead{February 14, 2017}
\lfoot{}
\cfoot{}
\rfoot{Page\ \thepage\ of\ \pageref{LastPage}}

\linespread{1.1}
\setlength{\parindent}{0pt}

\newcommand\blankpage{
    \thispagestyle{empty}
    \addtocounter{page}{-1}
    \newpage}
\renewcommand\footrulewidth{0.4pt}

\newcommand{\define}[1]{\underline{\textbf{#1}}}
\newcommand{\pn}[1]{\left( #1 \right)}
\newcommand{\hmap}[3][\varphi]{%
    #1:#2\rightarrow#3}

\DeclareMathOperator{\lcm}{lcm}

\mdfdefinestyle{mdgreenbox}{%
	roundcorner=10pt,
	linewidth=1pt,
	skipabove=12pt,
	innerbottommargin=9pt,
	skipbelow=2pt,
	linecolor=PineGreen,
	nobreak=true,
	backgroundcolor=SeaGreen!5,
}

\declaretheoremstyle[
headfont=\bfseries\color{Emerald},
mdframed={style=mdgreenbox},
headpunct={\\[3pt]},
postheadspace={0pt},
]{thmgreenbox}

\mdfdefinestyle{mdbluebox}{%
	roundcorner=10pt,
	linewidth=1pt,
	skipabove=12pt,
	innerbottommargin=9pt,
	skipbelow=2pt,
	linecolor=MidnightBlue,
	nobreak=true,
	backgroundcolor=Cyan!5,
}

\declaretheoremstyle[
headfont=\bfseries\color{NavyBlue},
mdframed={style=mdbluebox},
headpunct={\\[3pt]},
postheadspace={0pt},
]{thmbluebox}

\theoremstyle{definition}
\declaretheorem[name=\color{Emerald}\bfseries Definition, numbered=no]{defi}
\declaretheorem[style=thmgreenbox, name=Definition, numbered=no]{bdefi}
\declaretheorem[name=\color{NavyBlue}\bfseries Theorem, numbered=no]{thm}
\declaretheorem[style=thmbluebox, name=Theorem, numbered=no]{bthm}
\declaretheorem[name=\color{Cerulean}\bfseries Proof, numbered=no]{pf}
\declaretheorem[name=\color{Dandelion}\bfseries Example, numbered=no]{ex}
\declaretheorem[name=\color{Plum}\bfseries Proposition, numbered=no]{prop}
\declaretheorem[name=\color{Blue}\bfseries Note, numbered=no]{note}

\begin{document}
\begin{center}
    \LARGE{Abstract Algebra Notes}
\end{center}

\begin{defi}
    A \define{map} $f:A \rightarrow B$ is a subset $f\subset A\times B$ such that for all $a\in A$, there exists a $b\in B$ such that $b$ is unique with $(a,b)\in f$.
\end{defi}

\begin{defi}
    We write $f(a)=b$ if $(a,b)\in f$. $A$ is the \define{domain} of $f$ and $B$ is the \define{codomain}.
\end{defi}

\begin{defi}
    A \define{binary operation} on $A$ is a map $\star:A\times A\rightarrow A$ such that $\star(a_1,a_2)=a_1\star a_2$ for $a_1,a_2\in A$.
\end{defi}

\begin{defi}
    A binary operation $\star$ is \define{associative} on $A$ if for all $a,b,c\in A$, $a\star(b\star c)=(a\star b)\star c$.
\end{defi}

\begin{defi}
    An element $e\in A$ is an \define{identity} element of $\star$ if for each $a\in A$, $e\star a=a\star e=a$.
\end{defi}

\begin{defi}
    An element $a\in A$ has an \define{inverse} under $\star$ if there exists a $b\in A$ such that $a\star b=b\star a=e$.
\end{defi}

\begin{defi}
    A set $A$ with an associative binary operation $\star$ is a \define{group} if $A$ has an identity element under $\star$ and every $a\in A$ has an inverse.
\end{defi}

\begin{bdefi}
    A group is a pair $(G,\star)$ where $G$ is a set and $\star$ is a binary operation on $G$ such that
    \begin{enumerate}
        \item For all $a,b,c\in A$, $a\star(b\star c)=(a\star b)\star c$.
        \item There exists an $e\in G$ such that $a\star e=e\star a=a$ for all $a\in G$.
        \item For all $a\in G$, there exists a $b\in G$ such that $a\star b=b\star a=e$.
    \end{enumerate}
\end{bdefi}

\begin{defi}
    A group $(G,\star)$ is \define{abelian} or commutative if for all $g,h\in G$, $g\star h=h\star g$.
\end{defi}

\begin{thm}
    Let $(G,\star)$ be a group.
    \begin{enumerate}
        \item $e$ is unique.
        \item $g^{-1}$ is unique.
        \item $\forall g\in G, \pn{g^{-1}}^{-1}=g$.
        \item $\forall g,h\in G, (g\star h)^{-1}=h^{-1}\star g^{-1}$.
    \end{enumerate}
\end{thm}

\begin{pf}\begin{samepage}
    We may prove each part separately.
    \begin{enumerate}
        \item Suppose $e,e'$ are identity elements. Then for all $a\in G$,
            \begin{align*}
                a\star e&=e\star a=a\tag*{(i)}\\
                a\star e'&=e'\star a=a\tag*{(ii)}
            \end{align*}
            By (i), $e'=e\star e'$ and by (ii), $e=e\star e'$. Therefore, $e=e'$.
        \item Supposed $a\star b=b\star a=e$, then
            \begin{align*}
                b&=b\star e\\
                 &=b\star(a\star a^{-1})\\
                 &=(b\star a)\star a^{-1}\\
                 &=e\star a^{-1}\\
                 &=a^{-1}
            \end{align*}
            Thus, $b=a^{-1}$.
        \item $g^{-1}\star \pn{g^{-1}}^{-1}=e=g^{-1}\star g$. By (ii), $g=\pn{g^{-1}}^{-1}$.
        \item Consider $(a\star b)\star(b^{-1}\star a^{-1})$.
            \begin{align*}
                (a\star b)\star(b^{-1}\star a^{-1})&=a\star(b\star b^{-1})\star a^{-1}\\
                                                   &=a\star e\star a^{-1}\\
                                                   &=a\star a^{-1}\\
                                                   &=e
            \end{align*}
            Thus, $(b^{-1}\star a^{-1})=(a\star b)^{-1}$.
    \end{enumerate}
\end{samepage}\end{pf}

\begin{defi}
    Let $[n]=\{1,2,\ldots,n\}$. The \define{symmetric group} denoted $S_n$ of degree $n$ is the set of all bijections on $[n]$ under the operation of composition.
    $$S_n=\{\sigma:[n]\rightarrow[n]\mid\sigma\text{ is a bijection}\}$$
\end{defi}

\begin{defi}
    The \define{order} of $(G,\star)$ is the number of elements in $G$ denoted $|G|$.
\end{defi}

\begin{defi}
    Let $n\ge 2$. The \define{dihedral group} of index $n$ is the group of all symmetries of a regular polygon $P_n$ with $n$ vertices in the Euclidean plane.
\end{defi}

Symmetries of $P_n$ consist of rotations and reflections.\\

Choose a vertex $v$. Let $L_0$ be the line from the center of $P_n$ through $v$. Let $L_k$ be $L_0$ rotated by $\frac{\pi k}{n}$ for $1\le k\le n$. Let $\sigma_k$ be a reflection about $L_k$. Let $\rho_k$ be a rotation about $\frac{2\pi k}{n}$, $1\le k\le n$.

\end{document}
