\documentclass[11pt,letterpaper]{jacky}
\usepackage[margin=1in,headheight=14pt]{geometry}
\usepackage{amsfonts, amsmath, amssymb, enumerate, fancyhdr, gensymb, lastpage, mathtools}
\usepackage{tikz-cd}

\pagestyle{fancy}
\lhead{Jacky Lee}
\chead{Abstract Algebra Notes}
\rhead{February 14, 2017}
\lfoot{}
\cfoot{}
\rfoot{Page\ \thepage\ of\ \pageref{LastPage}}

\linespread{1.1}
\setlength{\parindent}{0pt}

\newcommand\blankpage{
    \thispagestyle{empty}
    \addtocounter{page}{-1}
    \newpage}
\renewcommand\footrulewidth{0.4pt}

\begin{document}
\begin{center}
    \vspace*{20pt}
    \LARGE{Scrawlings of the MagiKarp}
\end{center}

\begin{defi}
    A \define{map} $f:A \rightarrow B$ is a subset $f\subset A\times B$ such that for all $a\in A$, there exists a $b\in B$ such that $b$ is unique with $(a,b)\in f$.
\end{defi}

\begin{defi}
    We write $f(a)=b$ if $(a,b)\in f$. $A$ is the \define{domain} of $f$ and $B$ is the \define{codomain}.
\end{defi}

\begin{defi}
    A \define{binary operation} on $A$ is a map $\star:A\times A\rightarrow A$ such that $\star(a_1,a_2)=a_1\star a_2$ for $a_1,a_2\in A$.
\end{defi}

\begin{defi}
    A binary operation $\star$ is \define{associative} on $A$ if for all $a,b,c\in A$, $a\star(b\star c)=(a\star b)\star c$.
\end{defi}

\begin{defi}
    An element $e\in A$ is an \define{identity} element of $\star$ if for each $a\in A$, $e\star a=a\star e=a$.
\end{defi}

\begin{defi}
    An element $a\in A$ has an \define{inverse} under $\star$ if there exists a $b\in A$ such that $a\star b=b\star a=e$.
\end{defi}

\begin{defi}
    A set $A$ with an associative binary operation $\star$ is a \define{group} if $A$ has an identity element under $\star$ and every $a\in A$ has an inverse.
\end{defi}

\begin{bdefi}
    A group is a pair $(G,\star)$ where $G$ is a set and $\star$ is a binary operation on $G$ such that
    \begin{enumerate}
        \item For all $a,b,c\in A$, $a\star(b\star c)=(a\star b)\star c$.
        \item There exists an $e\in G$ such that $a\star e=e\star a=a$ for all $a\in G$.
        \item For all $a\in G$, there exists a $b\in G$ such that $a\star b=b\star a=e$.
    \end{enumerate}
\end{bdefi}

\begin{defi}
    A group $(G,\star)$ is \define{abelian} or commutative if for all $g,h\in G$, $g\star h=h\star g$.
\end{defi}

\begin{bthm}
    Let $(G,\star)$ be a group.
    \begin{enumerate}
        \item $e$ is unique.
        \item $g^{-1}$ is unique.
        \item $\forall g\in G, \pn{g^{-1}}^{-1}=g$.
        \item $\forall g,h\in G, (g\star h)^{-1}=h^{-1}\star g^{-1}$.
    \end{enumerate}
\end{bthm}

\begin{bpf}
    We may prove each part separately.
    \begin{enumerate}
        \item Suppose $e,e'$ are identity elements. Then for all $a\in G$,
            \begin{align*}
                a\star e&=e\star a=a\tag*{(i)}\\
                a\star e'&=e'\star a=a\tag*{(ii)}
            \end{align*}
            By (i), $e'=e\star e'$ and by (ii), $e=e\star e'$. Therefore, $e=e'$.
        \item Supposed $a\star b=b\star a=e$, then
            \begin{align*}
                b&=b\star e\\
                 &=b\star(a\star a^{-1})\\
                 &=(b\star a)\star a^{-1}\\
                 &=e\star a^{-1}\\
                 &=a^{-1}
            \end{align*}
            Thus, $b=a^{-1}$.
        \item $g^{-1}\star \pn{g^{-1}}^{-1}=e=g^{-1}\star g$. By (ii), $g=\pn{g^{-1}}^{-1}$.
        \item Consider $(a\star b)\star(b^{-1}\star a^{-1})$.
            \begin{align*}
                (a\star b)\star(b^{-1}\star a^{-1})&=a\star(b\star b^{-1})\star a^{-1}\\
                                                   &=a\star e\star a^{-1}\\
                                                   &=a\star a^{-1}\\
                                                   &=e
            \end{align*}
            Thus, $(b^{-1}\star a^{-1})=(a\star b)^{-1}$.
    \end{enumerate}
\end{bpf}

\begin{defi}
    Let $[n]=\{1,2,\ldots,n\}$. The \define{symmetric group} denoted $S_n$ of degree $n$ is the set of all bijections on $[n]$ under the operation of composition.
    $$S_n=\{\sigma:[n]\rightarrow[n]\mid\sigma\text{ is a bijection}\}$$
\end{defi}

\begin{defi}
    The \define{order} of $(G,\star)$ is the number of elements in $G$ denoted $|G|$.
\end{defi}

\begin{defi}
    Let $n\ge 2$. The \define{dihedral group} of index $n$ is the group of all symmetries of a regular polygon $P_n$ with $n$ vertices in the Euclidean plane.
\end{defi}

Symmetries of $P_n$ consist of rotations and reflections.\\

Choose a vertex $v$. Let $L_0$ be the line from the center of $P_n$ through $v$. Let $L_k$ be $L_0$ rotated by $\frac{\pi k}{n}$ for $1\le k\le n$. Let $\sigma_k$ be a reflection about $L_k$. Let $\rho_k$ be a rotation about $\frac{2\pi k}{n}$, $1\le k\le n$.

\begin{defi}
    A subset $S\subseteq G$ of a group $(G,\star)$ is a set of \define{generators}, denoted $\langle S\rangle=G$, if and only if every element of $G$ can be written as a finite product of elements of $S$ and their inverses.
\end{defi}

\begin{defi}
    Any equation satisfied by generators is called a \define{relation}.
\end{defi}

\begin{defi}
    A \define{presentation} of $G$, denoted $\langle S\mid R\rangle$, is a set of generators of $G$ and relations such that any other relation can be derived by those given.
\end{defi}

\begin{ex}
    $$D_{2n}=\langle r,s\mid r^n=s^2=1,rs=sr^{-1}\rangle$$
\end{ex}

\begin{defi}
    The cycles $\sigma=(\sigma_1\ \sigma_2\ \ldots\ \sigma_n)$ and $\tau=(\tau_1\ \tau_2\ \ldots\ \tau_n)$ are \define{disjoint} if $\sigma_i\ne\tau_j$ for $1\le i\le n$ and $1\le j\le m$.
\end{defi}

\begin{defi}
    A cycle of length 2 is called a \define{transposition}.
\end{defi}

\begin{defi}
    An expression of the form $(a_1\ a_2\ \ldots\ a_m)$ is called a \define{cycle of length m} or an \define{m-cycle}.
\end{defi}

\begin{prop}
    Let $\alpha=(a_1\ a_2\ \ldots\ a_m)$ and $\beta=(b_1\ b_2\ \ldots\ b_n)$. If $a_i\ne b_j$ for any $i,j$, then $\alpha\beta=\beta\alpha$.
\end{prop}

\begin{prop}
    Every permutation can be written as a product of disjoint cycles.
\end{prop}

\begin{prop}
    A cycle of length $n$ has order $n$.
\end{prop}

\begin{prop}
    Let $\alpha_1,\alpha_2,\ldots,\alpha_n$ be disjoint cycles. Then,
    $$|\alpha_1\alpha_2\ldots\alpha_n|=\lcm(|\alpha_1|,|\alpha_2|,\ldots,|\alpha_n|)$$
\end{prop}

\begin{prop}
    Every permutation is $S_n$ is a product of 2-cycles (which are not necessarily disjoint).
\end{prop}

\begin{prop}
    If $\alpha=\beta_1\beta_2\ldots\beta_r=\gamma_1\gamma_2\ldots\gamma_s$ where $\beta_i,\gamma_j$ are transpositions, then $r$ and $s$ have the same parity.
\end{prop}

\begin{defi}
    If $r$ and $s$ are both odd, $\alpha$ is called an \define{odd permutation}. If $r$ and $s$ are both even, $\alpha$ is called an \define{even permutation}.
\end{defi}

\begin{defi}
    The set of even permutations in $S_n$ form a group called the \define{alternating group}, denoted $A_n$.
\end{defi}

\begin{note}
    $|A_n|=\frac{n!}{2}$ for $n>1$.
\end{note}

\begin{bdefi}
    Let $(G,\star)$ and $(G',\ast)$ be groups. A map of sets $\hmap{G}{G'}$ is a \define{group homomorphism} if for all $a,b\in G$,
    $$\varphi(a\star b)=\varphi(a)\ast\varphi(b)$$
\end{bdefi}

\begin{ex}
    The following are two very simple examples of homomorphisms.\\\\
    Trivial Homomorphism
    $$\hmap{G}{G'},\varphi(g)=e,\forall g\in G$$
    Identity Homomorphism
    $$\hmap{G}{G'},\varphi(g)=g,\forall g\in G$$
\end{ex}

\begin{defi}
    If $\hmap{G}{G'}$ is a homomorphism, the \define{domain} of $\varphi$ is $\text{Dom}(\varphi)=G$, the \define{codomain} of $\varphi$ is $\text{Codom}(\varphi)=G'$, the \define{range} or \define{image} of $\varphi$ is $\varphi(G)=\{\varphi(g):g\in G\}\subseteq G'$ denoted $\text{Range}(\varphi)$ or $\im$.
\end{defi}

\begin{bdefi}
    A homomorphism which is bijective is called an \define{isomorphism}.
\end{bdefi}

$\hmap{G}{G'}$ is an isomorphism if and only if there exists $\hmap[\psi]{G'}{G}$ such that $\psi$ is a homomorphism and $\varphi\circ\psi=1_{G'}$, $\psi\circ\varphi=1_{G}$, i.e. $\psi$ is an inverse homomorphism to $\varphi$. We say $G$ is isomorphic to $G'$ by $G\cong G'$ or $\phi:G\xrightarrow{\sim}G'$.

\begin{bdefi}
    Let $(G,\star)$ be a group. A subset $H\subseteq G$ is a \define{subgroup} if $(H,\ast)$ is also a group.
\end{bdefi}

If $H\ne\emptyset$ and $H\subseteq G$, $H\le G$ or $H$ is a subgroup of $G$ if and only if
\begin{enumerate}
    \item $H$ is closed under $\star$ ($\forall h_1,h_2\in H$,\ $h_1\star h_2\in H$).
    \item $H$ is closed under inverses ($h\in H\Rightarrow h^{-1}\in H$).
\end{enumerate}

\begin{note}
    The following is notation for arbitrary and abelian groups.
    \begin{center}
        $x\star y\rightarrow xy$ for arbitrary $G$, $x+y$ for abelian $G$\\
        $e\rightarrow1$ for arbitrary $G$, 0 for abelian $G$
    \end{center}
\end{note}

For an arbitrary subset $A\subseteq G$, and $g\in G$,
$$gA=\{ga:a\in A\}\hspace{50pt}Ag=\{ag:a\in A\}\hspace{50pt}gAg^{-1}=\{gag^{-1}:a\in A\}$$

\begin{bthm}[Subgroup Criterion]
    Let $\emptyset\ne H\subseteq G$, $H\le G$ if and only if $\forall x,y\in H$, $xy^{-1}\in H$.
\end{bthm}

\begin{defi}
    Let $A\subseteq G$ be any subset. The \define{centralizer} of $A$ in $G$ is $C_G(A)=\{g\in G:gag^{-1}=a\}$ and it is the set of elements in $G$ which commute with all elements of $A$.
\end{defi}

\begin{prop}
    $C_G(A)\le G$
\end{prop}

\begin{pf}
    First we show that the centralizer is not empty. $1a=a1=a$, $\forall a\in A$ $\Rightarrow$ $1\in C_G(A)$ $\Rightarrow$ $C_G(A)\ne0$ so the centralizer of $A$ is not empty. Let $x,y\in C_G(A)$. We want to show that $xy^{-1}\in C_G(A)$ or that $xy^{-1}\in C_G(A)$. We do this by showing that $\pn{xy^{-1}}a\pn{xy^{-1}}^{-1}=a$.
    \begin{align*}
        \pn{xy^{-1}}a\pn{xy^{-1}}^{-1}&=xy^{-1}ayx^{-1}\\
        &=x\pn{y^{-1}ay}x^{-1}\\
        &=xax^{-1}\tag*{($y\in C_G(A)$)}\\
        &=a\tag*{($x\in C_G(A)$)}
    \end{align*}
    Since this subset satisfies the Subgroup Criterion, the centralizer $C_G(A)$ is a subgroup of $G$.
\end{pf}

\begin{defi}
    The \define{center} of a group $G$ is denoted $Z(G)=\{g\in G:gx=xg,\ \forall x\in G\}$. $Z(G)=C_G(G)\le G$. $Z(G)$ is the set of elements of $G$ which commute with all elements in $G$. If $G$ is abelian, $Z(G)=G$.
\end{defi}

\begin{defi}
    The \define{normalizer} of $A$ in $G$ is $N_G(A)=\{g\in G:gAg^{-1}=A\}$ or $\{g\in G:gag^{-1}=a'\in A\}$.
\end{defi}

\begin{prop}
    $C_G(A)\le N_G(A)\le G$
\end{prop}

\begin{defi}
    A \define{group action} of a group $G$ on a set $A$ is a map $G\times A\rightarrow A$ such that $(g_1g_2)\cdot a=g_1\cdot\pn{g_2\cdot a}$, $\forall g_1,g_2\in G$, $\forall a\in A$ and $1\cdot a=a$, $\forall a\in A$. It is denoted $G\circlearrowleft A$.
\end{defi}

\begin{defi}
    Suppose $G\circlearrowleft A$, the stabilizer of $a\in A$ in $G$ is $G_a=\{g\in G:g\cdot a=a\}$. $G_a\le G$.
\end{defi}

\begin{bdefi}
    An \define{equivalence relation} $\mathcal{E}$ on a set $S$ is a subset $\mathcal{E}\subseteq S\times S$ which is reflexive, symmetric, and transitive. We write $(a,b)\in\mathcal{E}\Leftrightarrow a\mathrel\mathcal{E}b$ or $a\sim b$.
    \begin{enumerate}
        \item $a\sim a$
        \item $a\sim b\Leftrightarrow b\sim a$
        \item $a\sim b$, $b\sim c\Rightarrow a\sim c$
    \end{enumerate}
\end{bdefi}

\begin{defi}
    The \define{equivalence class} of $a\in S$ is $[a]=\{b\in S:a\sim b\}$
\end{defi}

\begin{defi}
    The \define{quotient set} of $S$ under $\sim$ is $S$/$\sim=\{[a]:a\in S\}$.
\end{defi}

\begin{ex}
    $\mathbb{Q}=\{(a,b)\in\mathbb{Z}\times\mathbb{Z}:b\ne 0\}$/$\sim$, $(a,b)\sim(c,d)\Rightarrow ad=bc$.
\end{ex}

\begin{defi}
    The quotient set comes equipped with the \define{projection map} $\pi:S\rightarrow S$/$\sim$ where $a\mapsto[a]=\pi(a)$. This map is surjective by definition.
\end{defi}

\begin{bdefi}
    A group $G'$ is a \define{quotient group} of a group $G$ if
    \begin{enumerate}
        \item $G'=G$/$\sim$, $G'$ is the quotient set of $G$ under an equivalence relation $\sim$.
        \item The projection map $\hmap[\pi]{G}{G'}=G$/$\sim$ is a group homomorphism.
    \end{enumerate}
\end{bdefi}

\begin{defi}
    Let $\hmap{G}{G'}$ be a homomorphism and let $g'\in G'$. The \define{fiber} over $g'$ is $\varphi^{-1}(g')=\{g\in G:\varphi(g)=g'\}$.
\end{defi}

\begin{bprop}
    All quotient groups come from subgroups.
\end{bprop}

\begin{bpf}
    Let $\hmap{G}{G'}$ be a homomorphism, then $\varphi$ induces an equivalence relation on $G$. Let $x\sim y\Leftrightarrow\varphi(x)=\varphi(y)$. But $\varphi$ is a group homomorphism, so $\varphi(x)=\varphi(y)\Leftrightarrow\varphi(x)\varphi(y)^{-1}=1_{G'}\Leftrightarrow\varphi(x)\varphi(y^{-1})=1\Leftrightarrow\varphi(xy^{-1})=1$. So $x\sim y\Leftrightarrow\varphi(xy^{-1})=1$. Let $K=\{g\in G:\varphi(g)=1\}$. Then $x\sim y\Leftrightarrow xy^{-1}\in K$. Recall $K=\ker\le G$.\\\\
    Let $G'$ be a quotient group of $G$. Then $x\sim y\Leftrightarrow[x]=[y]\Leftrightarrow\pi(x)=\pi(y)$ where $\hmap[\pi]{G}{G'}$ is the projection. But $\pi(x)=\pi(y)\Leftrightarrow xy^{-1}\in\ker$.
\end{bpf}

\begin{defi}
    The \define{right coset} of a subgroup $H$ of a group $G$ by the element $x\in G$ is $Hx=\{hx:h\in H\}$. The \define{left coset}, denoted $xH$ is denoted similarly.
\end{defi}

\begin{prop}
    Let $\hmap{G}{G'}$ be a homomorphism and $K=\ker$. Then $xKx^{-1}\subseteq K$, $\forall x\in G$.
\end{prop}

\begin{pf}
    We must show $\varphi(xkx^{-1})=1_{G'}$ for $x\in G$, $k\in K$. Then, $\varphi(xkx^{-1})=\varphi(x)\varphi(k)\varphi(x^{-1})=\varphi(x)\varphi(x)^{-1}=1_{G'}$.
\end{pf}

\begin{bdefi}
    The subgroup $N\le G$ is \define{normal} if $xNx^{-1}\subseteq N$ for all $x\in G$. It is denoted $N\unlhd G$.
\end{bdefi}

\begin{prop}
    $\ker\unlhd G$ for any homomorphism $\hmap{G}{G'}$.
\end{prop}

\begin{bthm}
    Let $N\le G$. Then the following are equivalent.
    \begin{enumerate}
        \item $N\unlhd G$ ($xNx^{-1}\subseteq N,\ \forall x\in G$)
        \item $xNx^{-1}=N$
        \item $xN=Nx$
        \item $\forall x,y\in G,\ xy^{-1}\in N\Leftrightarrow y^{-1}x\in N$
    \end{enumerate}
\end{bthm}

\begin{bpf}
    $(1)\Rightarrow(2)$ Assume $\forall x\in G$, $xNx^{-1}\subseteq N$. We want to show $xNx^{-1}=N$. We do this by showing $N\subseteq xNx^{-1}$. Let $x\in G$, $n_0\in N$. We show $n_0\in xNx^{-1}$. Note that $x\in G\Rightarrow x^{-1}\in G$. Thus, $x^{-1}N\pn{x^{-1}}^{-1}\subseteq N$ since $N\unlhd G$. Thus there exists $n$ such that $x^{-1}nx=n_1\in N$. $n_0=x\pn{x^{-1}n_0x}x^{-1}=xn_1x^{-1}\in xNx^{-1}$.\\\\
    $(3)\Rightarrow(4)$ Assume $\forall x\in G$, $xN=Nx$. Let $x,y\in G$. We want to show $xy^{-1}\in N\Leftrightarrow y^{-1}x\in N$. So we must show this is true in both directions. Suppose $xy^{-1}\in N$. Then there exists an $n_1\in N$ such that $xy^{-1}=n_1$. Thus, $x=n_1y\in Ny=yN$ by assumption. So $x\in yN$. Thus there exists $n_2\in N$ such that $x=yn_2\Rightarrow y^{-1}x=n_2\in N$. Thus, $xy^{-1}\in N\Rightarrow y^{-1}x\in N$. Similarly, $y^{-1}x\in N\Rightarrow xy^{-1}\in N$.
\end{bpf}

\begin{prop}
    Let $H\le G$. Then, $x\sim y\Leftrightarrow y^{-1}x\in H$ is an equivalence relation on $G$.
\end{prop}

\begin{pf}
    We want to show $\sim$ is reflexive, symmetric, and transitive.
    \begin{enumerate}
        \item $x\sim x$: $x^{-1}x=1\in H$
        \item $x\sim y\Rightarrow y\sim x$: $x\sim y\Leftrightarrow y^{-1}x\in H\Rightarrow x^{-1}y\in H\Leftrightarrow y\sim x$
        \item $x\sim y$, $y\sim z\Rightarrow x\sim z$: $y^{-1}x\in H$, $z^{-1}y\in H\Rightarrow(z^{-1}y)(y^{-1}x)=z^{-1}x\in H\Leftrightarrow x\sim z$
    \end{enumerate}
    Thus, $\sim$ is an equivalence relation on $G$.
\end{pf}

Any subgroup gives an equivalence relation.

\begin{defi}
    An equivalence relation on a set $S$ is the same as a \define{partition} of $S$. $P=\{A_1,A_2,\ldots\}$, $A_i\subseteq S$ such that $S\cup_{i\in\mathbb{N}}A_i$, $A_i\cap A_j=\emptyset$, $i\ne j$. $a\sim b\Leftrightarrow a,b\in A_i$.
\end{defi}

\begin{prop}
    For $H\le G$, $x\sim y\Leftrightarrow y^{-1}x\in H\Leftrightarrow xH=yH\ (Hx=Hy)$.
\end{prop}

\begin{pf}
    Suppose $y^{-1}x\in H$. We want to show that $xH=yH$ or $xH\subseteq yH$ and $yH\subseteq xH$. $y^{-1}x\in H$ implies that there exists a $h_1\in H$ such that $y^{-1}x=h_1$. Thus, $x=yh_1\Rightarrow x\in yH$. $y^{-1}x\in H\Leftrightarrow x^{-1}y\in H$ which implies that there exists a $h_2\in H$ such that $x^{-1}y=h_2\Rightarrow y=xh_2\in xH$.
\end{pf}

\begin{note}
    $[x]=xH$.
\end{note}

\begin{prop}
    For $N\le G$, let $G/N=\{xN:x\in G\}$. Define $xN\cdot yN=(xy)N$. Then $G/N$ is a group if and only if $N\unlhd G$.
\end{prop}

$G/N=G$/$\sim$ ($x\sim y\Leftrightarrow xN=yN$)\\

Every quotient group is $G/N$ for some $N$.\\

$\pi:G\rightarrow G$/$\sim$, $\ker[(\pi)]\unlhd G$, $G$/$\sim\ =G/\ker[(\pi)]$.

\begin{prop}
    If $H\le G$ and $G$ is abelian, then $H\unlhd G$.
\end{prop}

If $G$ is a group and $\sim$ is an equivalence relation on $G$, then the quotient set $G$/$\sim$ is a quotient group if and only if the projection map $\pi:G\rightarrow G$/$\sim$, $\pi(x)=[x]$ is a homomorphism.\\

If $N\unlhd G$, then $G/N$ is a quotient group, where $G/N=\{xN:x\in G\}$ and $xN\cdot yN=(xy)N$.\\

These notions of quotient groups are equivalent.

\begin{prop}
    If $\sim$ is an equivalence relation and $G$/$\sim$ is a quotient group, then there exists a homomorphism $\pi:G\rightarrow G$/$\sim$ and $\ker[(\pi)]\unlhd G$.
\end{prop}

\begin{pf}
    $x\sim y\Leftrightarrow \pi(x)=\pi(y)\Leftrightarrow \pi(y^{-1}x)=1\Leftrightarrow y^{-1}x\in\ker[(\pi)]\Leftrightarrow x\ker[(\pi)]=y\ker[(\pi)]$.
\end{pf}

If $N\unlhd G$, define $x\sim y\Leftrightarrow xN=yN\Leftrightarrow y^{-1}x\in N$. Then, $G$/$\sim=G/N$, $[x]=xN$, $\hmap[\pi]{G}{G/N}$, $\pi(x)=xN$, $\ker[(\pi)]=N$.

\begin{prop}
    Every subgroup of an abelian group is a normal subgroup.
\end{prop}

\begin{defi}
    $S^n\subseteq\mathbb{R}^{n+1}$, $S^n=\{(x_1,x_2,\ldots,x_{n+1}):\sum x_i^2=1\}$
\end{defi}

For $H\le G$, the relation $x\sim y\Leftrightarrow xH=yH\Leftrightarrow y^{-1}x\in H$ is an equivalence relation and thus partitions $G$ into equivalence classes.

$$G=\bigcup\limits_{x\in G}[x],\ [x]\cap[y]=\emptyset,\ [x]\ne[y]$$
$$G=\bigcup\limits_{x\in G}xH,\ xH\cap yH=\emptyset,\ x\nsim y$$

\begin{prop}
    Let $H\le G$. The number of right cosets of $H$ equals the number of left cosets of $H$.
\end{prop}

\begin{pf}
    Let $R=\{Hx:x\in G\}$ and $L=\{xH:x\in G\}$. We construct a bijection $L\rightarrow R$. Define $\hmap[f]{R}{L}$ by $f(Hx)=x^{-1}H$, and define $\hmap[g]{L}{R}$ by $g(xH)=Hx^{-1}$. Then $f$ and $g$ are mutually inverse. Hence $R\leftrightarrow L$.
\end{pf}

\begin{defi}
    The number of distinct left cosets of $H$ in $G$ is called the \define{index} of $H$ in $G$, and is denoted $[G:H]$.
\end{defi}

\begin{bthm}[Lagrange's Theorem]
    If $H$ is a subgroup of $G$, $|G|=|H|[G:H]$.
\end{bthm}

\begin{cor}
    In a finite group, the order of every element divides the order of the group.
\end{cor}

\begin{cor}
    A group of prime order is cyclic.
\end{cor}

\begin{cor}
    Let $G$ be a finite group and let $a\in G$. Then, $a^{|G|}=1$.
\end{cor}

Let $\hmap{G}{G'}$ be a homomorphism. How far is $\varphi$ from an isomorphism? How can $\varphi$ fail to be an isomorphism?
\begin{enumerate}
    \item $\varphi$ could fail to be injective. $(\ker\ne\{1\})$
    \item $\varphi$ could fail to be surjective.
\end{enumerate}

\begin{bthm}[First Isomorphism Theorem]
    Let $\hmap{G}{G'}$ be a homomorphism. Then $\ker\unlhd G$, $\im\le G'$ and
    $$G/\ker\cong\im$$
\end{bthm}

\begin{prop}
    There exists an isomorphism $\hmap[\theta]{G/\ker}{\im}$ such that
    \begin{center}
        \begin{tikzcd}[column sep={8em,between origins}, row sep=6em, every matrix/.append style={name=m}, execute at end picture={
                    \draw [<-] ([xshift=-4em, yshift=2.25em]m-2-2.north) arc[start angle=-90,delta angle=270,radius=0.25cm];
            }]
            G \arrow[r, "\varphi"] \arrow[d, "\pi"] & G' \arrow[d, hookleftarrow, "\iota"] \\
            G/\ker \arrow[r, "\theta"] & \im
        \end{tikzcd}
    \end{center}
    The curved arrow in the middle means the diagram is commutative, i.e. $\varphi=\iota\cdot\theta\cdot\pi$. The curved arrow means it is injective.
\end{prop}

\begin{pf}
    Define $\hmap[\theta]{G/\ker}{\im}$ by $\theta(x\ker)=\varphi(x)$.\\\\
    First we show that $\theta$ is well-defined. Suppose $x\ker=y\ker$. Then,
    \begin{align*}
        x\ker=y\ker&\Leftrightarrow y^{-1}x\ker=\ker\\
                   &\Leftrightarrow y^{-1}x\in\ker\\
                   &\Leftrightarrow \varphi(y^{-1}x)=1\\
                   &\Leftrightarrow \varphi(y)^{-1}\varphi(x)=1\\
                   &\Leftrightarrow \varphi(x)=\varphi(y)\\
                   &\Leftrightarrow \theta(x\ker)=\theta(y\ker)
    \end{align*}
    Thus, $\theta$ is well-defined.\\\\
    Then, we show that $\theta$ is a homomorphism. Let $K=\ker$.
    \begin{align*}
        \theta(xKyK)&=\theta(xyK)\\
                    &=\varphi(xy)\\
                    &=\varphi(x)\varphi(y)\\
                    &=\theta(xK)\theta(yK)
    \end{align*}
    Thus, $\theta$ is a homomorphism.\\\\
    Then, we show that $\theta$ is injective.
    \begin{align*}
        \theta(xK)=\theta(yK)&\Leftrightarrow\varphi(x)=\varphi(y)\\
                             &\Leftrightarrow\varphi(y)^{-1}\varphi(x)=1\\
                             &\Leftrightarrow\varphi(y^{-1}x)=1\\
                             &\Leftrightarrow y^{-1}x\in K\\
                             &\Leftrightarrow xK=yK
    \end{align*}
    Thus, $\theta$ is injective.\\\\
    Then, we show that $\theta$ is surjective. Let $y\in\im$. There exists $xK\in G/K$ such that $\theta(xK)=y$. We know there exists an $x\in G$ such that $\varphi(x)=y$. $\theta(xK)=\varphi(x)=y$. Thus, $\theta$ is surjective and $\theta$ is an isomorphism.
\end{pf}

\begin{prop}
    Let $a\in G$. If $|a|=\infty$, then $\langle a\rangle\cong(\mathbb{Z},+)$. If $|a|=n$, then $\langle a\rangle=\mathbb{Z}_n=\mathbb{Z}/n\mathbb{Z}$.
\end{prop}

\begin{pf}
    Consider $\mathbb{Z}\xrightarrow{\pi} G$ defined by $\pi(k)=a^k$.
\end{pf}

\begin{defi}
    Let $(A,\star)$ and $(B,\ast)$ be groups. The \define{direct product} or \define{direct sum} of $A$ and $B$ is $A\oplus B=\{(a,b):a\in A,b\in B\}$ where $(a_1,b_1)\cdot(a_2,b_2)=(a_1\star a_2,b_1\ast b_2)\in A\oplus B$.
\end{defi}

\begin{defi}
    In a group $G$, define $a\sim b\Leftrightarrow\exists x\in G$ such that $b=xax^{-1}$. This is an equivalence relation and $a$ and $b$ are \define{conjugates}.
\end{defi}

\begin{defi}
    For any $x\in G$, the \define{inner automorphism} of $G$ induced by $x$ is $\hmap[T_x]{G}{G}$ defined by $T_x(g)=xgx^{-1}$.
\end{defi}

\begin{defi}
    The set of all inner automorphisms of $G$ is a group, called the \define{inner automorphism group}, and is denoted $\text{Inn}(G)=\{\hmap[T_x]{G}{G}\mid x\in G\}$.
\end{defi}

\begin{prop}
    $G/Z(G)\cong\text{Inn}(G)$
\end{prop}

\begin{pf}
    Consider $\hmap[\psi]{G}{\text{Inn}(G)}$ defined by $x\mapsto T_x$. Then, $\psi$ is surjective, i.e. $\im[(\psi)]=\text{Inn}(G)$. We then determine the kernel of the homomorphism.
    \begin{align*}
        \ker[(\psi)]&=\{x\in G:\psi(x)=1_G\}\\
                    &=\{x\in G:T_x(g)=g,\ \forall g\in G\}\\
                    &=\{x\in G:xgx^{-1}=g,\ \forall g\in G\}\\
                    &=\{x\in G:xg=gx,\ \forall g\in G\}\\
                    &=Z(G)
    \end{align*}
    By the first isomorphism theorem, $G/Z(G)\cong\text{Inn}(G)$.
\end{pf}

\begin{bthm}[Third Isomorphism Theorem]
    Let $G$ be a group. Let $A\unlhd G$, $B\unlhd G$. If $A\subseteq B$, then $A\unlhd B$, $B/A\unlhd G/A$, and
    $$(G/A)/(B/A)\cong(G/B)$$
\end{bthm}

\begin{bpf}
    First we establish $A\unlhd B$. $A\le B$ because $A\le G$ and $A\subseteq B$.
    $$A\unlhd B\Leftrightarrow bAb^{-1}\subseteq A,\ \forall b\in B$$
    $$A\unlhd G\Leftrightarrow xAx^{-1}\subseteq A,\ \forall x\in G$$
    But $B\subseteq G$ so $b\in G$. Thus, $bAb^{-1}\subseteq A$, $\forall b\in B$ and $A\unlhd B$. Thus, $A\unlhd B$ and we may construct $B/A$.\\\\
    We first show $B/A\le G/A$. It is closed under multiplication since $(b_1A)(b_2A)=(b_1b_2)A\in B/A$ because $B$ is a group. It is also closed under inverses since $(bA)^{-1}=b^{-1}A\in B/A$.\\\\
    We then show $B/A\unlhd G/A$ by showing $x(B/A)x^{-1}\subseteq B/A$, $\forall x\in G/A$. Let $x\in G/A\Leftrightarrow yA$, $y\in G$. We want to show $(yA)(B/A)(yA)^{-1}\subseteq B/A$. Let $z\in(yA)(B/A)(yA)^{-1}$. Then, there exist $a_1,a_2\in A$, $b_1\in B$ such that
    \begin{align*}
        z&=(ya_1)(b_1A)(y^{-1}a_2)\\
         &=y(a_1b_1)Ay^{-1}a_2\\
         &=y(a_1b_1)y^{-1}Aa_2
    \end{align*}
    We know $a_2\in A\Rightarrow Aa_2=A$ and $A\subseteq B\Rightarrow a_1\in A\subseteq B\Rightarrow a_1\in A\Rightarrow a_1b_1\in B$. Thus, there exists $b_2\in B$ such that $a_1b_1=b_2$. We substitute these in to get
    $$z=yb_2y^{-1}A$$
    We know $B\unlhd G\Rightarrow yBy^{-1}\subseteq B$. Thus, there exists a $b_3\in B$ such that $yb_2y^{-1}=b_3\in B$. We then get $z=b_3A$. Since $z=b_3A\in B/A$, $B/A\unlhd G/A$.\\\\
    Now we prove $(G/A)/(B/A)\cong(G/B)$. We define the homomorphism $\hmap[\omega]{G/A}{G/B}$ such that $\omega(xA)=xB$. We show that $\omega$ is well-defined. If $xA=yA$, then
    \begin{align*}
        xA=yA&\Leftrightarrow y^{-1}x\in A\subseteq B\\
             &\Rightarrow y^{-1}x\in B\\
             &\Leftrightarrow xB=yB\\
             &\Leftrightarrow\omega(xA)=\omega(yA)
    \end{align*}
    We may then determine the kernel and image of the homomorphism.
    $$\im[(\omega)]=\{xB:x\in G\}=G/B$$
    $$\ker[(\omega)]=\{xA:\omega(xA)=B\}=\{xA:xB=B\}=\{xA:x\in B\}=B/A$$
    By the first isomorphism theorem, $(G/A)/\ker[(\omega)]\cong\im[(\omega)]$ so $(G/A)/(B/A)\cong(G/B)$.
\end{bpf}

\begin{prop}
    There is an isomorphism $\hmap[\theta]{(G/A)/(B/A)}{G/B}$ such that this diagram commutes.
    \begin{center}
        \begin{tikzcd}[column sep={6em,between origins}, row sep=huge]
            & G/A \arrow[dl, leftarrow, "\pi"] \arrow[dr, "\sigma"] \arrow[dd, "\omega"]& \\
            G \arrow[dr, "\rho"] & & (G/A)/(B/A) \arrow[dl, "\theta"] \\
            & G/B &
        \end{tikzcd}
    \end{center}
\end{prop}

\begin{bthm}[Second Isomorphism Theorem]
    Let $G$ be a group, $A\le G$, and $N\unlhd G$. Then $AN\le G$, $N\unlhd AN$, and $A\cap N\unlhd A$. Also,
    $$(AN)/N\cong A/(A\cap N)$$
\end{bthm}

\begin{bpf}
    Let $\hmap{A}{AN/N}$ such that $a\mapsto aN$. Then by the first isomorphism theorem, $(AN)/N\cong A/(A\cap N)$.
\end{bpf}

\begin{ex}
    We look at an example of the third isomorphism theorem. Let $G=\mathbb{Z}$, $A=12\mathbb{Z}$, and $B=4\mathbb{Z}$. We observe that $A\unlhd B\unlhd G$ so the conditions for the third isomorphism theorem are satisfied.
    $$G/A=\mathbb{Z}/12\mathbb{Z}=\{0,1,\ldots,11\}(\text{mod }12)$$
    $$B/A=4\mathbb{Z}/12\mathbb{Z}=\{0,4,8\}(\text{mod }12)$$
    $$(G/A)/(B/A)=\{0,1,2,3\}(\text{mod }4)=\mathbb{Z}/4\mathbb{Z}$$
    $$(\mathbb{Z}/12\mathbb{Z})/(4\mathbb{Z}/12\mathbb{Z})\cong\mathbb{Z}/4\mathbb{Z}$$
\end{ex}

\begin{ex}
    We look at an example of the second isomorphism theorem. Let $G=\mathbb{Z}$, $N=12\mathbb{Z}$, and $A=8\mathbb{Z}$.
    $$A\cap N=\{0,\pn24,\pn48,\ldots\}=24\mathbb{Z}$$
    $$AN=\{0,\pn4,\pn8,\ldots\}=4\mathbb{Z}$$
    $$AN/A=4\mathbb{Z}/12\mathbb{Z}=\{0,4,8\}(\text{mod }12)$$
    $$A/(A\cap N)=8\mathbb{Z}/24\mathbb{Z}=\{0,8,16\}(\text{mod }24)$$
    $$AN/N\cong\mathbb{Z}/3\mathbb{Z}\cong A/(A\cap N)$$
\end{ex}

\end{document}
