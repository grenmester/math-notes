\documentclass[11pt,letterpaper]{book}
\usepackage[margin=1in,headheight=14pt]{geometry}
\usepackage{amsfonts, amsmath, amssymb, enumerate, fancyhdr, gensymb, lastpage, mathtools}
\usepackage[usenames, dvipsnames, svgnames]{xcolor}
\usepackage{amsthm, thmtools}
\usepackage[framemethod=TikZ]{mdframed}
\usepackage{mathpazo}
\usepackage{tikz-cd}

\pagestyle{fancy}
\lhead{Jacky Lee}
\chead{Abstract Algebra Notes}
\rhead{February 14, 2017}
\lfoot{}
\cfoot{}
\rfoot{Page\ \thepage\ of\ \pageref{LastPage}}

\linespread{1.1}
\setlength{\parindent}{0pt}

\newcommand\blankpage{
    \thispagestyle{empty}
    \addtocounter{page}{-1}
    \newpage}
\renewcommand\footrulewidth{0.4pt}

\newcommand{\define}[1]{\underline{\textbf{#1}}}
\newcommand{\pn}[1]{\left( #1 \right)}
\newcommand{\hmap}[3][\varphi]{%
    #1:#2\rightarrow#3}
\renewcommand{\ker}[1][(\varphi)]{\text{Ker}#1}
\newcommand{\im}[1][(\varphi)]{\text{Im}#1}

\DeclareMathOperator{\lcm}{lcm}

\mdfdefinestyle{mdgreenbox}{%
	roundcorner=10pt,
	linewidth=1pt,
	skipabove=12pt,
	innerbottommargin=9pt,
	skipbelow=2pt,
	linecolor=PineGreen,
	nobreak=true,
	backgroundcolor=SeaGreen!5,
}

\declaretheoremstyle[
headfont=\bfseries\color{Emerald},
mdframed={style=mdgreenbox},
headpunct={\\[3pt]},
postheadspace={0pt},
]{thmgreenbox}

\mdfdefinestyle{mdcyanbox}{%
	roundcorner=10pt,
	linewidth=1pt,
	skipabove=12pt,
	innerbottommargin=9pt,
	skipbelow=2pt,
	linecolor=RoyalBlue,
	nobreak=true,
	backgroundcolor=CornflowerBlue!5,
}

\declaretheoremstyle[
headfont=\bfseries\color{Cerulean},
mdframed={style=mdcyanbox},
headpunct={\\[3pt]},
postheadspace={0pt},
]{thmcyanbox}

\mdfdefinestyle{mdbluebox}{%
	roundcorner=10pt,
	linewidth=1pt,
	skipabove=12pt,
	innerbottommargin=9pt,
	skipbelow=2pt,
	linecolor=MidnightBlue,
	nobreak=true,
	backgroundcolor=BlueViolet!5,
}

\declaretheoremstyle[
headfont=\bfseries\color{NavyBlue},
mdframed={style=mdbluebox},
headpunct={\\[3pt]},
postheadspace={0pt},
]{thmbluebox}

\mdfdefinestyle{mdpurplebox}{%
	roundcorner=10pt,
	linewidth=1pt,
	skipabove=12pt,
	innerbottommargin=9pt,
	skipbelow=2pt,
	linecolor=DarkOrchid,
	nobreak=true,
	backgroundcolor=Orchid!5,
}

\declaretheoremstyle[
headfont=\bfseries\color{Plum},
mdframed={style=mdpurplebox},
headpunct={\\[3pt]},
postheadspace={0pt},
]{thmpurplebox}

\theoremstyle{definition}
\declaretheorem[name=\color{Emerald}\bfseries Definition, numbered=no]{defi}
\declaretheorem[style=thmgreenbox, name=Definition, numbered=no]{bdefi}
\declaretheorem[name=\color{NavyBlue}\bfseries Theorem, numbered=no]{thm}
\declaretheorem[style=thmbluebox, name=Theorem, numbered=no]{bthm}
\declaretheorem[name=\color{Cerulean}\bfseries Proof, numbered=no]{pf}
\declaretheorem[style=thmcyanbox, name=Proof, numbered=no]{bpf}
\declaretheorem[name=\color{Dandelion}\bfseries Example, numbered=no]{ex}
\declaretheorem[name=\color{Plum}\bfseries Proposition, numbered=no]{prop}
\declaretheorem[style=thmpurplebox, name=Proposition, numbered=no]{bprop}
\declaretheorem[name=\color{Blue}\bfseries Note, numbered=no]{note}
\declaretheorem[name=\color{Gray}\bfseries Corollary, numbered=no]{cor}

\begin{document}
\begin{center}
    \vspace*{20pt}
    \LARGE{Scrawlings of the MagiKarp}
\end{center}

\begin{defi}
    A \define{map} $f:A \rightarrow B$ is a subset $f\subset A\times B$ such that for all $a\in A$, there exists a $b\in B$ such that $b$ is unique with $(a,b)\in f$.
\end{defi}

\begin{defi}
    We write $f(a)=b$ if $(a,b)\in f$. $A$ is the \define{domain} of $f$ and $B$ is the \define{codomain}.
\end{defi}

\begin{defi}
    A \define{binary operation} on $A$ is a map $\star:A\times A\rightarrow A$ such that $\star(a_1,a_2)=a_1\star a_2$ for $a_1,a_2\in A$.
\end{defi}

\begin{defi}
    A binary operation $\star$ is \define{associative} on $A$ if for all $a,b,c\in A$, $a\star(b\star c)=(a\star b)\star c$.
\end{defi}

\begin{defi}
    An element $e\in A$ is an \define{identity} element of $\star$ if for each $a\in A$, $e\star a=a\star e=a$.
\end{defi}

\begin{defi}
    An element $a\in A$ has an \define{inverse} under $\star$ if there exists a $b\in A$ such that $a\star b=b\star a=e$.
\end{defi}

\begin{defi}
    A set $A$ with an associative binary operation $\star$ is a \define{group} if $A$ has an identity element under $\star$ and every $a\in A$ has an inverse.
\end{defi}

\begin{bdefi}
    A group is a pair $(G,\star)$ where $G$ is a set and $\star$ is a binary operation on $G$ such that
    \begin{enumerate}
        \item For all $a,b,c\in A$, $a\star(b\star c)=(a\star b)\star c$.
        \item There exists an $e\in G$ such that $a\star e=e\star a=a$ for all $a\in G$.
        \item For all $a\in G$, there exists a $b\in G$ such that $a\star b=b\star a=e$.
    \end{enumerate}
\end{bdefi}

\begin{defi}
    A group $(G,\star)$ is \define{abelian} or commutative if for all $g,h\in G$, $g\star h=h\star g$.
\end{defi}

\begin{bthm}
    Let $(G,\star)$ be a group.
    \begin{enumerate}
        \item $e$ is unique.
        \item $g^{-1}$ is unique.
        \item $\forall g\in G, \pn{g^{-1}}^{-1}=g$.
        \item $\forall g,h\in G, (g\star h)^{-1}=h^{-1}\star g^{-1}$.
    \end{enumerate}
\end{bthm}

\begin{bpf}
    We may prove each part separately.
    \begin{enumerate}
        \item Suppose $e,e'$ are identity elements. Then for all $a\in G$,
            \begin{align*}
                a\star e&=e\star a=a\tag*{(i)}\\
                a\star e'&=e'\star a=a\tag*{(ii)}
            \end{align*}
            By (i), $e'=e\star e'$ and by (ii), $e=e\star e'$. Therefore, $e=e'$.
        \item Supposed $a\star b=b\star a=e$, then
            \begin{align*}
                b&=b\star e\\
                 &=b\star(a\star a^{-1})\\
                 &=(b\star a)\star a^{-1}\\
                 &=e\star a^{-1}\\
                 &=a^{-1}
            \end{align*}
            Thus, $b=a^{-1}$.
        \item $g^{-1}\star \pn{g^{-1}}^{-1}=e=g^{-1}\star g$. By (ii), $g=\pn{g^{-1}}^{-1}$.
        \item Consider $(a\star b)\star(b^{-1}\star a^{-1})$.
            \begin{align*}
                (a\star b)\star(b^{-1}\star a^{-1})&=a\star(b\star b^{-1})\star a^{-1}\\
                                                   &=a\star e\star a^{-1}\\
                                                   &=a\star a^{-1}\\
                                                   &=e
            \end{align*}
            Thus, $(b^{-1}\star a^{-1})=(a\star b)^{-1}$.
    \end{enumerate}
\end{bpf}

\begin{defi}
    Let $[n]=\{1,2,\ldots,n\}$. The \define{symmetric group} denoted $S_n$ of degree $n$ is the set of all bijections on $[n]$ under the operation of composition.
    $$S_n=\{\sigma:[n]\rightarrow[n]\mid\sigma\text{ is a bijection}\}$$
\end{defi}

\begin{defi}
    The \define{order} of $(G,\star)$ is the number of elements in $G$ denoted $|G|$.
\end{defi}

\begin{defi}
    Let $n\ge 2$. The \define{dihedral group} of index $n$ is the group of all symmetries of a regular polygon $P_n$ with $n$ vertices in the Euclidean plane.
\end{defi}

Symmetries of $P_n$ consist of rotations and reflections.\\

Choose a vertex $v$. Let $L_0$ be the line from the center of $P_n$ through $v$. Let $L_k$ be $L_0$ rotated by $\frac{\pi k}{n}$ for $1\le k\le n$. Let $\sigma_k$ be a reflection about $L_k$. Let $\rho_k$ be a rotation about $\frac{2\pi k}{n}$, $1\le k\le n$.

\begin{defi}
    A subset $S\subseteq G$ of a group $(G,\star)$ is a set of \define{generators}, denoted $\langle S\rangle=G$, if and only if every element of $G$ can be written as a product of elements of $S$ and their inverses.
\end{defi}

\begin{defi}
    Any equation satisfied by generators is called a \define{relation}.
\end{defi}

\begin{defi}
    A \define{presentation} of $G$, denoted $\langle S\mid R\rangle$, is a set of generators of $G$ and relations such that any other relation can be derived by those given.
\end{defi}

\begin{ex}
    $$D_{2n}=\langle r,s\mid r^n=s^2=1,rs=sr^{-1}\rangle$$
\end{ex}

\begin{defi}
    The cycles $\sigma=(\sigma_1\ \sigma_2\ \ldots\ \sigma_n)$ and $\tau=(\tau_1\ \tau_2\ \ldots\ \tau_n)$ are \define{disjoint} if $\sigma_i\ne\tau_j$ for $1\le i\le n$ and $1\le j\le m$.
\end{defi}

\begin{defi}
    A cycle of length 2 is called a \define{transposition}.
\end{defi}

\begin{defi}
    An expression of the form $(a_1\ a_2\ \ldots\ a_m)$ is called a \define{cycle of length m} or an \define{m-cycle}.
\end{defi}

\begin{prop}
    Let $\alpha=(a_1\ a_2\ \ldots\ a_m)$ and $\beta=(b_1\ b_2\ \ldots\ b_n)$. If $a_i\ne b_j$ for any $i,j$, then $\alpha\beta=\beta\alpha$.
\end{prop}

\begin{prop}
    Every permutation can be written as a product of disjoint cycles.
\end{prop}

\begin{prop}
    A cycle of length $n$ has order $n$.
\end{prop}

\begin{prop}
    Let $\alpha_1,\alpha_2,\ldots,\alpha_n$ be disjoint cycles. Then,
    $$|\alpha_1\alpha_2\ldots\alpha_n|=\lcm(|\alpha_1|,|\alpha_2|,\ldots,|\alpha_n|)$$
\end{prop}

\begin{prop}
    Every permutation is $S_n$ is a product of 2-cycles (which are not necessarily disjoint).
\end{prop}

\begin{prop}
    If $\alpha=\beta_1\beta_2\ldots\beta_r=\gamma_1\gamma_2\ldots\gamma_s$ where $\beta_i,\gamma_j$ are transpositions, then $r$ and $s$ have the same parity.
\end{prop}

\begin{defi}
    If $r$ and $s$ are both odd, $\alpha$ is called an \define{odd permutation}. If $r$ and $s$ are both even, $\alpha$ is called an \define{even permutation}.
\end{defi}

\begin{defi}
    The set of even permutations in $S_n$ form a group called the \define{alternating group}, denoted $A_n$.
\end{defi}

\begin{note}
    $|A_n|=\frac{n!}{2}$ for $n>1$.
\end{note}

\begin{bdefi}
    Let $(G,\star)$ and $(G',\ast)$ be groups. A map of sets $\hmap{G}{G'}$ is a \define{group homomorphism} if for all $a,b\in G$,
    $$\varphi(a\star b)=\varphi(a)\ast\varphi(b)$$
\end{bdefi}

\begin{ex}
    The following are two very simple examples of homomorphisms.\\\\
    Trivial Homomorphism
    $$\hmap{G}{G'},\varphi(g)=e,\forall g\in G$$
    Identity Homomorphism
    $$\hmap{G}{G'},\varphi(g)=g,\forall g\in G$$
\end{ex}

\begin{defi}
    If $\hmap{G}{G'}$ is a homomorphism, the \define{domain} of $\varphi$ is $\text{Dom}(\varphi)=G$, the \define{codomain} of $\varphi$ is $\text{Codom}(\varphi)=G'$, the \define{range} or \define{image} of $\varphi$ is $\varphi(G)=\{\varphi(g):g\in G\}\subseteq G'$ denoted $\text{Range}(\varphi)$ or $\im$.
\end{defi}

\begin{bdefi}
    A homomorphism which is bijective is called an \define{isomorphism}.
\end{bdefi}

$\hmap{G}{G'}$ is an isomorphism if and only if there exists $\hmap[\psi]{G'}{G}$ such that $\psi$ is a homomorphism and $\varphi\circ\psi=1_{G'}$, $\psi\circ\varphi=1_{G}$, i.e. $\psi$ is an inverse homomorphism to $\varphi$. We say $G$ is isomorphic to $G'$ by $G\cong G'$ or $\phi:G\xrightarrow{\sim}G'$.

\begin{bdefi}
    Let $(G,\star)$ be a group. A subset $H\subseteq G$ is a \define{subgroup} if $(H,\ast)$ is also a group.
\end{bdefi}

If $H\ne\emptyset$ and $H\subseteq G$, $H\le G$ or $H$ is a subgroup of $G$ if and only if
\begin{enumerate}
    \item $H$ is closed under $\star$ ($\forall h_1,h_2\in H$,\ $h_1\star h_2\in H$).
    \item $H$ is closed under inverses ($h\in H\Rightarrow h^{-1}\in H$).
\end{enumerate}

\begin{note}
    The following is notation for arbitrary and abelian groups.
    \begin{center}
        $x\star y\rightarrow xy$ for arbitrary $G$, $x+y$ for abelian $G$\\
        $e\rightarrow1$ for arbitrary $G$, 0 for abelian $G$
    \end{center}
\end{note}

For an arbitrary subset $A\subseteq G$, and $g\in G$,
$$gA=\{ga:a\in A\}\hspace{50pt}Ag=\{ag:a\in A\}\hspace{50pt}gAg^{-1}=\{gag^{-1}:a\in A\}$$

\begin{bthm}[Subgroup Criterion]
    Let $\emptyset\ne H\subseteq G$, $H\le G$ if and only if $\forall x,y\in H$, $xy^{-1}\in H$.
\end{bthm}

\begin{defi}
    Let $A\subseteq G$ be any subset. The \define{centralizer} of $A$ in $G$ is $C_G(A)=\{g\in G:gag^{-1}=a\}$ and it is the set of elements in $G$ which commute with all elements of $A$.
\end{defi}

\begin{prop}
    $C_G(A)\le G$
\end{prop}

\begin{pf}
    First we show that the centralizer is not empty. $1a=a1=a$, $\forall a\in A$ $\Rightarrow$ $1\in C_G(A)$ $\Rightarrow$ $C_G(A)\ne0$ so the centralizer of $A$ is not empty. Let $x,y\in C_G(A)$. We want to show that $xy^{-1}\in C_G(A)$ or that $xy^{-1}\in C_G(A)$. We do this by showing that $\pn{xy^{-1}}a\pn{xy^{-1}}^{-1}=a$.
    \begin{align*}
        \pn{xy^{-1}}a\pn{xy^{-1}}^{-1}&=xy^{-1}ayx^{-1}\\
        &=x\pn{y^{-1}ay}x^{-1}\\
        &=xax^{-1}\tag*{($y\in C_G(A)$)}\\
        &=a\tag*{($x\in C_G(A)$)}
    \end{align*}
    Since this subset satisfies the Subgroup Criterion, the centralizer $C_G(A)$ is a subgroup of $G$.
\end{pf}

\begin{defi}
    The \define{center} of a group $G$ is denoted $Z(G)=\{g\in G:gx=xg,\ \forall x\in G\}$. $Z(G)=C_G(G)\le G$. $Z(G)$ is the set of elements of $G$ which commute with all elements in $G$. If $G$ is abelian, $Z(G)=G$.
\end{defi}

\begin{defi}
    The \define{normalizer} of $A$ in $G$ is $N_G(A)=\{g\in G:gAg^{-1}=A\}$ or $\{g\in G:gag^{-1}=a'\in A\}$.
\end{defi}

\begin{prop}
    $C_G(A)\le N_G(A)\le G$
\end{prop}

\begin{defi}
    A \define{group action} of a group $G$ on a set $A$ is a map $G\times A\rightarrow A$ such that $(g_1g_2)\cdot a=g_1\cdot\pn{g_2\cdot a}$, $\forall g_1,g_2\in G$, $\forall a\in A$ and $1\cdot a=a$, $\forall a\in A$. It is denoted $G\circlearrowleft A$.
\end{defi}

\begin{defi}
    Suppose $G\circlearrowleft A$, the stabilizer of $a\in A$ in $G$ is $G_a=\{g\in G:g\cdot a=a\}$. $G_a\le G$.
\end{defi}

\begin{bdefi}
    An \define{equivalence relation} $\mathcal{E}$ on a set $S$ is a subset $\mathcal{E}\subseteq S\times S$ which is reflexive, symmetric, and transitive. We write $(a,b)\in\mathcal{E}\Leftrightarrow a\mathrel\mathcal{E}b$ or $a\sim b$.
    \begin{enumerate}
        \item $a\sim a$
        \item $a\sim b\Leftrightarrow b\sim a$
        \item $a\sim b$, $b\sim c\Rightarrow a\sim c$
    \end{enumerate}
\end{bdefi}

\begin{defi}
    The \define{equivalence class} of $a\in S$ is $[a]=\{b\in S:a\sim b\}$
\end{defi}

\begin{defi}
    The \define{quotient set} of $S$ under $\sim$ is $S$/$\sim=\{[a]:a\in S\}$.
\end{defi}

\begin{ex}
    $\mathbb{Q}=\{(a,b)\in\mathbb{Z}\times\mathbb{Z}:b\ne 0\}$/$\sim$, $(a,b)\sim(c,d)\Rightarrow ad=bc$.
\end{ex}

\begin{defi}
    The quotient set comes equipped with the \define{projection map} $\pi:S\rightarrow S$/$\sim$ where $a\mapsto[a]=\pi(a)$. This map is surjective by definition.
\end{defi}

\begin{bdefi}
    A group $G'$ is a \define{quotient group} of a group $G$ if
    \begin{enumerate}
        \item $G'=G$/$\sim$, $G'$ is the quotient set of $G$ under an equivalence relation $\sim$.
        \item The projection map $\hmap[\pi]{G}{G'}=G$/$\sim$ is a group homomorphism.
    \end{enumerate}
\end{bdefi}

\begin{defi}
    Let $\hmap{G}{G'}$ be a homomorphism and let $g'\in G'$. The \define{fiber} over $g'$ is $\varphi^{-1}(g')=\{g\in G:\varphi(g)=g'\}$.
\end{defi}

\begin{bprop}
    All quotient groups come from subgroups.
\end{bprop}

\begin{bpf}
    Let $\hmap{G}{G'}$ be a homomorphism, then $\varphi$ induces an equivalence relation on $G$. Let $x\sim y\Leftrightarrow\varphi(x)=\varphi(y)$. But $\varphi$ is a group homomorphism, so $\varphi(x)=\varphi(y)\Leftrightarrow\varphi(x)\varphi(y)^{-1}=1_{G'}\Leftrightarrow\varphi(x)\varphi(y^{-1})=1\Leftrightarrow\varphi(xy^{-1})=1$. So $x\sim y\Leftrightarrow\varphi(xy^{-1})=1$. Let $K=\{g\in G:\varphi(g)=1\}$. Then $x\sim y\Leftrightarrow xy^{-1}\in K$. Recall $K=\ker\le G$.\\\\
    Let $G'$ be a quotient group of $G$. Then $x\sim y\Leftrightarrow[x]=[y]\Leftrightarrow\pi(x)=\pi(y)$ where $\hmap[\pi]{G}{G'}$ is the projection. But $\pi(x)=\pi(y)\Leftrightarrow xy^{-1}\in\ker$.
\end{bpf}

\begin{defi}
    The \define{right coset} of a subgroup $H$ of a group $G$ by the element $x\in G$ is $Hx=\{hx:h\in H\}$. The \define{left coset}, denoted $xH$ is denoted similarly.
\end{defi}

\begin{prop}
    Let $\hmap{G}{G'}$ be a homomorphism and $K=\ker$. Then $xKx^{-1}\subseteq K$, $\forall x\in G$.
\end{prop}

\begin{pf}
    We must show $\varphi(xkx^{-1})=1_{G'}$ for $x\in G$, $k\in K$. Then, $\varphi(xkx^{-1})=\varphi(x)\varphi(k)\varphi(x^{-1})=\varphi(x)\varphi(x)^{-1}=1_{G'}$.
\end{pf}

\begin{bdefi}
    The subgroup $N\le G$ is \define{normal} if $xNx^{-1}\subseteq N$ for all $x\in G$. It is denoted $N\unlhd G$.
\end{bdefi}

\begin{prop}
    $\ker\unlhd G$ for any homomorphism $\hmap{G}{G'}$.
\end{prop}

\begin{bthm}
    Let $N\le G$. Then the following are equivalent.
    \begin{enumerate}
        \item $N\unlhd G$ ($xNx^{-1}\subseteq N,\ \forall x\in G$)
        \item $xNx^{-1}=N$
        \item $xN=Nx$
        \item $\forall x,y\in G,\ xy^{-1}\in N\Leftrightarrow y^{-1}x\in N$
    \end{enumerate}
\end{bthm}

\begin{bpf}
    $(1)\Rightarrow(2)$ Assume $\forall x\in G$, $xNx^{-1}\subseteq N$. We want to show $xNx^{-1}=N$. We do this by showing $N\subseteq xNx^{-1}$. Let $x\in G$, $n_0\in N$. We show $n_0\in xNx^{-1}$. Note that $x\in G\Rightarrow x^{-1}\in G$. Thus, $x^{-1}N\pn{x^{-1}}^{-1}\subseteq N$ since $N\unlhd G$. Thus there exists $n$ such that $x^{-1}nx=n_1\in N$. $n_0=x\pn{x^{-1}n_0x}x^{-1}=xn_1x^{-1}\in xNx^{-1}$.\\\\
    $(3)\Rightarrow(4)$ Assume $\forall x\in G$, $xN=Nx$. Let $x,y\in G$. We want to show $xy^{-1}\in N\Leftrightarrow y^{-1}x\in N$. So we must show this is true in both directions. Suppose $xy^{-1}\in N$. Then there exists an $n_1\in N$ such that $xy^{-1}=n_1$. Thus, $x=n_1y\in Ny=yN$ by assumption. So $x\in yN$. Thus there exists $n_2\in N$ such that $x=yn_2\Rightarrow y^{-1}x=n_2\in N$. Thus, $xy^{-1}\in N\Rightarrow y^{-1}x\in N$. Similarly, $y^{-1}x\in N\Rightarrow xy^{-1}\in N$.
\end{bpf}

\end{document}
