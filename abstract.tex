\documentclass[11pt,letterpaper,boxed]{hmcpset}
\usepackage[margin=1in,headheight=14pt]{geometry}
\usepackage{amsfonts, amsmath, amssymb, enumerate, fancyhdr, gensymb, lastpage, mathtools}
\usepackage[usenames, dvipsnames, svgnames]{xcolor}
\usepackage{amsthm, thmtools}
\usepackage[framemethod=TikZ]{mdframed}

\pagestyle{fancy}
\lhead{Jacky Lee}
\chead{Abstract Algebra Notes}
\rhead{February 14, 2017}
\lfoot{}
\cfoot{}
\rfoot{Page\ \thepage\ of\ \pageref{LastPage}}

\linespread{1.1}
\setlength{\parindent}{0pt}

\newcommand\blankpage{
    \thispagestyle{empty}
    \addtocounter{page}{-1}
    \newpage}
\renewcommand\footrulewidth{0.4pt}

\newcommand{\define}[1]{\underline{\textbf{#1}}}
\newcommand{\pn}[1]{\left( #1 \right)}

\mdfdefinestyle{mdgreenbox}{%
	roundcorner=10pt,
	linewidth=1pt,
	skipabove=12pt,
	innerbottommargin=9pt,
	skipbelow=2pt,
	linecolor=PineGreen,
	nobreak=true,
	backgroundcolor=SeaGreen!5,
}

\declaretheoremstyle[
headfont=\bfseries\color{Emerald},
mdframed={style=mdgreenbox},
headpunct={\\[3pt]},
postheadspace={0pt},
]{thmgreenbox}

\theoremstyle{definition}
\declaretheorem[name=\color{Emerald}\bfseries Definition, numbered=no]{defi}
%\declaretheorem[name=\color{ForestGreen}\bfseries Definition, numbered=no]{bdefi}
\declaretheorem[style=thmgreenbox, name=Definition, numbered=no]{bdefi}
\declaretheorem[name=\color{NavyBlue}\bfseries Theorem, numbered=no]{thm}
\declaretheorem[name=\color{Cerulean}\bfseries Proof, numbered=no]{pf}

\begin{document}
\begin{center}
    \LARGE{Abstract Algebra Notes}
\end{center}

\begin{defi}
    A \define{map} $f:A \rightarrow B$ is a subset $f\subset A\times B$ such that for all $a\in A$, there exists a $b\in B$ such that $b$ is unique with $(a,b)\in f$.
\end{defi}

\begin{defi}
    We write $f(a)=b$ if $(a,b)\in f$. $A$ is the \define{domain} of $f$ and $B$ is the \define{codomain}.
\end{defi}

\begin{defi}
    A \define{binary operation} on $A$ is a map $\star:A\times A\rightarrow A$ such that $\star(a_1,a_2)=a_1\star a_2$ for $a_1,a_2\in A$.
\end{defi}

\begin{defi}
    A binary operation $\star$ is \define{associative} on $A$ if for all $a,b,c\in A$, $a\star(b\star c)=(a\star b)\star c$.
\end{defi}

\begin{defi}
    An element $e\in A$ is an \define{identity} element of $\star$ if for each $a\in A$, $e\star a=a\star e=a$.
\end{defi}

\begin{defi}
    An element $a\in A$ has an \define{inverse} under $\star$ if there exists a $b\in A$ such that $a\star b=b\star a=e$.
\end{defi}

\begin{defi}
    A set $A$ with an associative binary operation $\star$ is a \define{group} if $A$ has an identity element under $\star$ and every $a\in A$ has an inverse.
\end{defi}

\begin{bdefi}
    A group is a pair $(G,\star)$ where $G$ is a set and $\star$ is a binary operation on $G$ such that
    \begin{enumerate}
        \item For all $a,b,c\in A$, $a\star(b\star c)=(a\star b)\star c$.
        \item There exists an $e\in G$ such that $a\star e=e\star a=a$ for all $a\in G$.
        \item For all $a\in G$, there exists a $b\in G$ such that $a\star b=b\star a=e$.
    \end{enumerate}
\end{bdefi}

\end{document}
