\documentclass{jacky}
\usepackage{tikz-cd} % Commutative diagrams

\name{Jacky Lee}
\notetitle{Category Theory Notes}
\notedate{\today}

\NewDocumentCommand{\dom}{o}{
  \IfNoValueTF{#1}
    {\text{dom}}
    {\text{dom }#1}
}
\NewDocumentCommand{\cod}{o}{
  \IfNoValueTF{#1}
    {\text{cod}}
    {\text{cod }#1}
}
\newcommand{\id}[1]{\text{id}_{#1}}
\newcommand{\cat}[1]{\text{\textbf{#1}}}

\begin{document}
\begin{center}
    \vspace*{20pt}
    \LARGE{Category Theory Notes}
\end{center}

\begin{defi}
    A \define{metagraph} consists of \define{objects} $a, b, c, \ldots$,
    \define{arrows} $f, g, h, \ldots$, and two operations as follows:
    \begin{itemize}
        \item \define{Domain}, which assigns to each arrow $f$ an object
            $a=\dom[f]$
        \item \define{Codomain}, which assigns to each arrow $f$ an object
            $b=\cod[f]$
    \end{itemize}
    These operations on $f$ are usually indicated by
    \[f:a\rightarrow b\hspace{10pt}\text{or}\hspace{10pt}a\xrightarrow{f}b\]
\end{defi}

\begin{defi}
    A \define{metacategory} is a metagraph with two additional operations:
    \begin{itemize}
        \item \define{Identity}, which assigns to each object $a$ an arrow
            $\id{a}=1_a:a\rightarrow a$
        \item \define{Composition}, which assigns to each pair $\langle g,f
            \rangle$ of arrows with $\dom[g]=\cod[f]$ an arrow $g\circ f$,
            called their \define{composite}, with $g\circ
            f:\dom[f]\rightarrow\cod[g]$
    \end{itemize}
    This operation may be pictured by the diagram
    \begin{center}
        \begin{tikzcd}
            A\arrow[r, "f"]\arrow[dr, "g\circ f", swap] & B\arrow[d, "g"]\\
            & C
        \end{tikzcd}
    \end{center}
    These operations in a metacategory are subject to the two following axioms:
    \begin{itemize}
        \item \define{Associativity}. For given objects and arrow in the
            configuration
            \[a\xrightarrow{f}b\xrightarrow{g}c\xrightarrow{k}d\]
            one always has the equality
            \[k\circ\left(g\circ f\right)=\left(k\circ g\right)\circ f\]
            This axiom is represented by the statement that the following
            diagram is commutative.
            \begin{center}
                \begin{tikzcd}[column sep=3cm, row sep=huge]
                    A\arrow[r, "k\circ (g\circ f) = (k\circ g)\circ f"]
                    \arrow[d, "f"']\arrow[dr, "g\circ f", pos=0.2, swap] & D \\
                    B\arrow[r, "g", swap]
                    \arrow[ur, "k\circ g", pos=0.8, swap, crossing over]
                    & C\arrow[u, "k"']
                \end{tikzcd}
            \end{center}
        \item \define{Unit law}. For all arrows $f:a\rightarrow b$ and
            $g:b\rightarrow c$, composition with the identity arrow $1_b$ gives
            \[1_b\circ f=f\hspace{10pt}\text{and}\hspace{10pt}g\circ1_b=g\]
            This axiom is represented by the statement that the following
            diagram is commutative.
            \begin{center}
                \begin{tikzcd}
                    A\arrow[r, "f"]\arrow[dr, "f", swap]
                    & B\arrow[d, "1_b"]\arrow[dr, "g"', swap]\\
                    & B\arrow[r, "g"'] & C
                \end{tikzcd}
            \end{center}
    \end{itemize}
\end{defi}

\begin{defi}
    A \define{directed graph} is a set $O$ of objects, a set $A$ of arrows, and
    two functions
    \[A\xrightarrow{\dom}O\hspace{50pt}A\xrightarrow{\cod}O\]
    The set of composable pairs of arrows is the set
    \[A\times_OA=\{\langle g,f\rangle\ |\ g,f\in A
        \text{ and }\dom[g]=\cod[f]\}\]
\end{defi}

\begin{bdefi}
    A \define{category} is a directed graph with two additional functions
    \[O\xrightarrow{\text{id}}A,\ A\times_OA\xrightarrow{\circ}A,\]
    \[c\mapsto\text{id}_c,\ \langle g,f\rangle\mapsto g\circ f\]
    called identity and composition (also written as $gf$), such that
    \[\dom(\id\ a)=a=\cod(\id\ a),\hspace{20pt}\dom(g\circ
    f)=\dom[f],\hspace{20pt}\cod(g\circ f)=\cod[g]\]
    for all objects $a\in O$ and all composable pairs of arrows $\langle
    g,f\rangle\in A\times_OA$, and such that the associativity and unit axioms
    hold.
\end{bdefi}

\begin{note}
    We usually drop the letters $A$ and $O$, and write
    \[c\in C\hspace{50pt}f\text{ in }C\]
    for ``$c$ is an object of $C$'' and ``$f$ is an arrow of $C$'',
    respectively.
\end{note}

\begin{defi}
    We write
    \[\hom(b,c)=\{f\ |\ f\text{ in }C,\dom[f]=b, \cod[f]=c\}\]
    for the set of arrows from $b$ to $c$.
\end{defi}

\begin{ex}
    The following are simple examples of categories.
    \begin{itemize}
        \item \textbf{0} is the empty category with no objects or arrows.
        \item \textbf{1} is the category with one object and one identity
            arrow.
        \item \textbf{2} is the category $A\rightarrow B$ with two objects and
            one non-identity arrow.
        \item \textbf{3} is the category with three objects whose non-identity
            arrows are arranged as in the following triangle.
            \begin{center}
                \begin{tikzcd}
                    A\arrow[r]\arrow[dr, swap] & B\arrow[d]\\
                    & C
                \end{tikzcd}
            \end{center}
        \item $\downdownarrows$ is the category $A\rightrightarrows B$ with two
            objects and two non-identity arrows. We call two such arrows
            \define{parallel arrows}.
    \end{itemize}
\end{ex}

\begin{bdefi}
    A \define{functor} is a morphism of categories. For categories $C$ and $B$,
    a functor $T:C\rightarrow B$ consists of two related functions: The
    \define{object function} $T$, which assigns to each object $c$ of $C$ an
    object $Tc$ of $B$ and the \define{arrow function} (also written $T$) which
    assigns to each arrow $f:c\rightarrow c'$ of $C$ an arrow $Tf:Tc\rightarrow
    Tc'$ of $B$, in such a way that
    \[T(1_c)=1_{Tc}\hspace{50pt} T(g\circ f)=Tg\circ Tf\]
\end{bdefi}

\begin{defi}
    The \define{forgetful functor} (or \define{underlying functor}) is a
    functor which ``forgets'' some of all of the structure of an algebraic
    object.
\end{defi}

\begin{ex}
    The functor $U: \cat{Rng}\rightarrow\cat{Grp}$ that assigns to each ring
    $R$ the set $UR$ of its elements and assigns to each morphism
    $f:R\rightarrow R'$ of rings the same function $f$, regarded just as a
    group homomorphism rather than a ring homomorphism (thus ``forgetting'' the
    multiplication).
\end{ex}

\begin{defi}
    The \define{composite} of two functors $T:C\rightarrow B$ and
    $S:B\rightarrow A$ is $S\circ T:C\rightarrow A$.
\end{defi}

\begin{defi}
    An \define{isomorphism} $T:C\rightarrow B$ of categories is a functor $T$
    from $C$ to $B$ which is a bijection, both on objects and on arrows.
    Alternatively, a functor $T:C\rightarrow B$ is an isomorphism if and only
    if there exists a functor $S:B\rightarrow C$ for which both composites
    $S\circ T$ and $T\circ S$ are identity functors. Then, $S$ is the
    \define{two-sided inverse} $S=T^{-1}$.
\end{defi}

\begin{defi}
    A functor $T:C\rightarrow B$ is \define{full} when to every pair $c, c'$ of
    objects of $C$ and to every arrow $g:Tc\rightarrow Tc'$ of $B$, there
    exists an arrow $f:c\rightarrow c'$ of $C$ with $g=Tf$.
\end{defi}

\begin{defi}
    A functor $T:C\rightarrow B$ is \define{faithful} when to every pair $c,
    c'$ of objects of $C$ and to every pair $f_1,f_2:c\rightarrow c'$ of
    parallel arrows of $C$, the equality $Tf_1=Tf_2:Tc\rightarrow Tc'$ implies
    $f_1=f_2$.
\end{defi}

For a pair of objects $c,c'\in C$, the arrow function of $T:C\rightarrow B$
assigns to each $f:c\rightarrow c'$ an arrow $Tf:Tc\rightarrow Tc'$ and so we
can define
\[T_{c,c'}:\hom(c,c')\rightarrow\hom(Tc,Tc'),\hspace{50pt}f\mapsto Tf\]
Then $T$ is full when every such function is surjective, and faithful when
every such function is injective. For a functor which is both full and
faithful, every such function is a bijection, but this need not mean that the
functor itself is an isomorphism of categories, for there may be objects of $B$
not in the image of $T$.

\begin{defi}
    A \define{subcategory} $S$ of a category $C$ is a collection of some of the
    objects and some of the arrows of $C$, which includes with each arrow $f$
    both the object $\dom[f]$ and the object $\cod[f]$, with each object $s$
    its identity arrow $1_s$ and with each pair of composable arrows
    $s\rightarrow s'\rightarrow s''$ their composite.
\end{defi}

\begin{defi}
    Given a subcategory $S$ of $C$, the \define{inclusion functor} is the
    functor $T:S\rightarrow C$ which sends each object and each arrow of $S$ to
    itself in $C$.
\end{defi}

\begin{defi}
    $S$ is a \define{full subcategory} of $C$ when the inclusion functor
    $T:S\rightarrow C$ is full.
\end{defi}

\begin{bdefi}
    Given two functors $S,T:C\rightarrow B$, a \define{natural transformation}
    $\tau:S\xrightarrow{\bullet}T$ is a function which assign to each object
    $c$ of $C$ an arrow $\tau_c=\tau c:Sc\rightarrow Tc$ of $B$ in such a way
    that every arrow $f:c\rightarrow c'$ in $C$ yields a diagram
    \begin{center}
        \begin{tikzcd}
            c\arrow[d, "f"] & Sc\arrow[r,"\tau c"]\arrow[d,"Sf"] &
            Tc\arrow[d,"Tf"]\\
            c'              & Sc'\arrow[r,"\tau c'"]             &
            Tc'
        \end{tikzcd}
    \end{center}
    which is commutative. When this holds, we say that $\tau_c:Sc\rightarrow Tc$
    is \define{natural} in $c$. We call $\tau a,\tau b,\tau c,\ldots$ the
    \define{components} of the natural transformation $\tau$.
\end{bdefi}

\begin{defi}
    A natural transformation $\tau$ with every component $\tau c$ invertible in
    $B$ is called a\\
    \define{natural equivalence} or a \define{natural
    isomorphism}. Symbolically, $\tau:S\cong T$.
\end{defi}

\end{document}
