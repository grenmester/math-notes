\documentclass[11pt,letterpaper]{jacky}
\usepackage[margin=1in,headheight=14pt]{geometry}
\usepackage{amsfonts, amsmath, amssymb, enumerate, fancyhdr, gensymb, lastpage,
mathtools, scrextend, tikz-cd}

\pagestyle{fancy}
\lhead{Jacky Lee}
\chead{Category Theory Notes}
\rhead{December 25, 2017}
\lfoot{}
\cfoot{}
\rfoot{Page\ \thepage\ of\ \pageref{LastPage}}

\linespread{1.1}
\setlength{\parindent}{0pt}

\newcommand\blankpage{
    \thispagestyle{empty}
    \addtocounter{page}{-1}
    \newpage}
\renewcommand\footrulewidth{0.4pt}

\begin{document}
\begin{center}
    \vspace*{20pt}
    \LARGE{Category Theory Notes}
\end{center}

\begin{defi}
    A \define{metagraph} consists of \define{objects} $a, b, c, \ldots$,
    \define{arrows} $f, g, h, \ldots$, and two operations as follows:
    \begin{itemize}
        \item \define{Domain}, which assigns to each arrow $f$ an object
            $a=\dom{f}$
        \item \define{Codomain}, which assigns to each arrow $f$ an object
            $b=\cod{f}$
    \end{itemize}
    These operations on $f$ are usually indicated by
    \[f:a\rightarrow b\hspace{10pt}\text{or}\hspace{10pt}a\xrightarrow{f}b\]
\end{defi}

\begin{defi}
    A \define{metacategory} is a metagraph with two additional operations:
    \begin{itemize}
        \item \define{Identity}, which assigns to each object $a$ an arrow
            $\id{a}=1_a:a\rightarrow a$
        \item \define{Composition}, which assigns to each pair $\langle g,f
            \rangle$ of arrows with $\dom{g}=\cod{f}$ an arrow $g\circ f$,
            called their \define{composite}, with $g\circ
            f:\dom{f}\rightarrow\cod{g}$
    \end{itemize}
    This operation may be pictured by the diagram
    \begin{center}
        \begin{tikzcd}
            A\arrow[r, "f"]\arrow[dr, "g\circ f", swap] & B\arrow[d, "g"]\\
            & C
        \end{tikzcd}
    \end{center}
    These operations in a metacategory are subject to the two following axioms:
    \begin{itemize}
        \item \define{Associativity}. For given objects and arrow in the
            configuration
            \[a\xrightarrow{f}b\xrightarrow{g}c\xrightarrow{k}d\]
            one always has the equality
            \[k\circ\left(g\circ f\right)=\left(k\circ g\right)\circ f\]
            This axiom is represented by the statement that the following
            diagram is commutative.
            \begin{center}
                \begin{tikzcd}[column sep=3cm, row sep=huge]
                    A\arrow[r, "k\circ (g\circ f) = (k\circ g)\circ f"]
                    \arrow[d, "f"']\arrow[dr, "g\circ f", pos=0.2, swap] & D \\
                    B\arrow[r, "g", swap]
                    \arrow[ur, "k\circ g", pos=0.8, swap, crossing over]
                    & C\arrow[u, "k"']
                \end{tikzcd}
            \end{center}
        \item \define{Unit law}. For all arrows $f:a\rightarrow b$ and
            $g:b\rightarrow c$, composition with the identity arrow $1_b$ gives
            \[1_b\circ f=f\hspace{10pt}\text{and}\hspace{10pt}g\circ1_b=g\]
            This axiom is represented by the statement that the following
            diagram is commutative.
            \begin{center}
                \begin{tikzcd}
                    A\arrow[r, "f"]\arrow[dr, "f", swap]
                    & B\arrow[d, "1_b"]\arrow[dr, "g"', swap]\\
                    & B\arrow[r, "g"'] & C
                \end{tikzcd}
            \end{center}
    \end{itemize}
\end{defi}
\end{document}
