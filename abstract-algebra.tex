\documentclass{mathnotes}
\usepackage{tikz-cd} % Commutative diagrams

\name{Jacky Lee}
\notetitle{Abstract Algebra Notes}
\notedate{\today}

\DeclareMathOperator{\lcm}{lcm}
\DeclareMathOperator{\im}{im}
\DeclareMathOperator{\Inn}{Inn}

\begin{document}
\begin{center}
  \vspace*{20pt}
  \LARGE{Abstract Algebra Notes}
\end{center}

\begin{defi}
  A \define{map} $f:A \rightarrow B$ is a subset $f\subset A\times B$ such that
  for all $a\in A$, there exists a $b\in B$ such that $b$ is unique with
  $(a,b)\in f$.
\end{defi}

\begin{defi}
  We write $f(a)=b$ if $(a,b)\in f$. $A$ is the \define{domain} of $f$ and $B$
  is the \define{codomain}.
\end{defi}

\begin{defi}
  A \define{binary operation} on $A$ is a map $\star:A\times A\rightarrow A$
  such that $\star(a_1,a_2)=a_1\star a_2$ for $a_1,a_2\in A$.
\end{defi}

\begin{defi}
  A binary operation $\star$ is \define{associative} on $A$ if for all
  $a,b,c\in A$, $a\star(b\star c)=(a\star b)\star c$.
\end{defi}

\begin{defi}
  An element $e\in A$ is an \define{identity} element of $\star$ if for each
  $a\in A$, $e\star a=a\star e=a$.
\end{defi}

\begin{defi}
  An element $a\in A$ has an \define{inverse} under $\star$ if there exists a
  $b\in A$ such that $a\star b=b\star a=e$.
\end{defi}

\begin{defi}
  A set $A$ with an associative binary operation $\star$ is a \define{group} if
  $A$ has an identity element under $\star$ and every $a\in A$ has an inverse.
\end{defi}

\begin{bdefi}
  A group is a pair $(G,\star)$ where $G$ is a set and $\star$ is a binary
  operation on $G$ such that
  \begin{enumerate}
    \item For all $a,b,c\in A$, $a\star(b\star c)=(a\star b)\star c$.
    \item There exists an $e\in G$ such that $a\star e=e\star a=a$ for all
      $a\in G$.
    \item For all $a\in G$, there exists a $b\in G$ such that $a\star b=b\star
      a=e$.
  \end{enumerate}
\end{bdefi}

\begin{defi}
  A group $(G,\star)$ is \define{abelian} or commutative if for all $g,h\in G$,
  $g\star h=h\star g$.
\end{defi}

\begin{bthm}
  Let $(G,\star)$ be a group.
  \begin{enumerate}
    \item $e$ is unique.
    \item $g^{-1}$ is unique.
    \item $\forall g\in G, \pn{g^{-1}}^{-1}=g$.
    \item $\forall g,h\in G, (g\star h)^{-1}=h^{-1}\star g^{-1}$.
  \end{enumerate}
\end{bthm}

\begin{bpf}
  We may prove each part separately.
  \begin{enumerate}
    \item Suppose $e,e'$ are identity elements. Then for all $a\in G$,
      \begin{align*}
        a\star e&=e\star a=a\tag*{(i)}\\
        a\star e'&=e'\star a=a\tag*{(ii)}
      \end{align*}
      By (i), $e'=e\star e'$ and by (ii), $e=e\star e'$. Therefore, $e=e'$.
    \item Supposed $a\star b=b\star a=e$, then
      \begin{align*}
        b&=b\star e\\
         &=b\star(a\star a^{-1})\\
         &=(b\star a)\star a^{-1}\\
         &=e\star a^{-1}\\
         &=a^{-1}
      \end{align*}
      Thus, $b=a^{-1}$.
    \item $g^{-1}\star \pn{g^{-1}}^{-1}=e=g^{-1}\star g$. By (ii),
      $g=\pn{g^{-1}}^{-1}$.
    \item Consider $(a\star b)\star(b^{-1}\star a^{-1})$.
      \begin{align*}
        (a\star b)\star(b^{-1}\star a^{-1})
        &=a\star(b\star b^{-1})\star a^{-1}\\
        &=a\star e\star a^{-1}\\
        &=a\star a^{-1}\\
        &=e
      \end{align*}
      Thus, $(b^{-1}\star a^{-1})=(a\star b)^{-1}$.
  \end{enumerate}
\end{bpf}

\begin{defi}
  Let $[n]=\{1,2,\ldots,n\}$. The \define{symmetric group} denoted $S_n$ of
  degree $n$ is the set of all bijections on $[n]$ under the operation of
  composition.
  $$S_n=\{\sigma:[n]\rightarrow[n]\mid\sigma\text{ is a bijection}\}$$
\end{defi}

\begin{defi}
  The \define{order} of $(G,\star)$ is the number of elements in $G$ denoted
  $|G|$.
\end{defi}

\begin{defi}
  Let $n\ge 2$. The \define{dihedral group} of index $n$ is the group of all
  symmetries of a regular polygon $P_n$ with $n$ vertices in the Euclidean
  plane.
\end{defi}

Symmetries of $P_n$ consist of rotations and reflections.\\

Choose a vertex $v$. Let $L_0$ be the line from the center of $P_n$ through
$v$. Let $L_k$ be $L_0$ rotated by $\frac{\pi k}{n}$ for $1\le k\le n$. Let
$\sigma_k$ be a reflection about $L_k$. Let $\rho_k$ be a rotation about
$\frac{2\pi k}{n}$, $1\le k\le n$.

\begin{defi}
  A subset $S\subseteq G$ of a group $(G,\star)$ is a set of
  \define{generators}, denoted $\langle S\rangle=G$, if and only if every
  element of $G$ can be written as a finite product of elements of $S$ and
  their inverses.
\end{defi}

\begin{defi}
  Any equation satisfied by generators is called a \define{relation}.
\end{defi}

\begin{defi}
  A \define{presentation} of $G$, denoted $\langle S\mid R\rangle$, is a set of
  generators of $G$ and relations such that any other relation can be derived
  by those given.
\end{defi}

\begin{ex}
  $$D_{2n}=\langle r,s\mid r^n=s^2=1,rs=sr^{-1}\rangle$$
\end{ex}

\begin{defi}
  The cycles $\sigma=(\sigma_1\ \sigma_2\ \ldots\ \sigma_n)$ and $\tau=(\tau_1\
  \tau_2\ \ldots\ \tau_n)$ are \define{disjoint} if $\sigma_i\ne\tau_j$ for
  $1\le i\le n$ and $1\le j\le m$.
\end{defi}

\begin{defi}
  A cycle of length 2 is called a \define{transposition}.
\end{defi}

\begin{defi}
  An expression of the form $(a_1\ a_2\ \ldots\ a_m)$ is called a \define{cycle
  of length m} or an \define{m-cycle}.
\end{defi}

\begin{prop}
  Let $\alpha=(a_1\ a_2\ \ldots\ a_m)$ and $\beta=(b_1\ b_2\ \ldots\ b_n)$. If
  $a_i\ne b_j$ for any $i,j$, then $\alpha\beta=\beta\alpha$.
\end{prop}

\begin{prop}
  Every permutation can be written as a product of disjoint cycles.
\end{prop}

\begin{prop}
  A cycle of length $n$ has order $n$.
\end{prop}

\begin{prop}
  Let $\alpha_1,\alpha_2,\ldots,\alpha_n$ be disjoint cycles. Then,
  $$|\alpha_1\alpha_2\ldots\alpha_n|=
  \lcm(|\alpha_1|,|\alpha_2|,\ldots,|\alpha_n|)$$
\end{prop}

\begin{prop}
  Every permutation is $S_n$ is a product of 2-cycles (which are not
  necessarily disjoint).
\end{prop}

\begin{prop}
  If $\alpha=\beta_1\beta_2\ldots\beta_r=\gamma_1\gamma_2\ldots\gamma_s$
  where $\beta_i,\gamma_j$ are transpositions, then $r$ and $s$ have the same
  parity.
\end{prop}

\begin{defi}
  If $r$ and $s$ are both odd, $\alpha$ is called an \define{odd
  permutation}. If $r$ and $s$ are both even, $\alpha$ is called an
  \define{even permutation}.
\end{defi}

\begin{defi}
  The set of even permutations in $S_n$ form a group called the
  \define{alternating group}, denoted $A_n$.
\end{defi}

\begin{note}
  $|A_n|=\frac{n!}{2}$ for $n>1$.
\end{note}

\begin{bdefi}
  Let $(G,\star)$ and $(G',\ast)$ be groups. A map of sets $\varphi:G\to G'$ is
  a \define{group homomorphism} if for all $a,b\in G$,
  $$\varphi(a\star b)=\varphi(a)\ast\varphi(b)$$
\end{bdefi}

\begin{ex}
  The following are two very simple examples of homomorphisms.\\\\
  \textbf{Trivial Homomorphism}
  $$\varphi:G\to G',\varphi(g)=e,\forall g\in G$$
  \textbf{Identity Homomorphism}
  $$\varphi:G\to G',\varphi(g)=g,\forall g\in G$$
\end{ex}

\begin{defi}
  If $\varphi:G\to G'$ is a homomorphism, the \define{domain} of $\varphi$ is
  $\text{Dom}(\varphi)=G$, the \define{codomain} of $\varphi$ is
  $\text{Codom}(\varphi)=G'$, the \define{range} or \define{image} of
  $\varphi$ is $\varphi(G)=\{\varphi(g):g\in G\}\subseteq G'$ denoted
  $\text{Range}(\varphi)$ or $\im\varphi$.
\end{defi}

\begin{bdefi}
  A homomorphism which is bijective is called an \define{isomorphism}.
\end{bdefi}

$\varphi:G\to G'$ is an isomorphism if and only if there exists $\psi:G'\to G$
such that $\psi$ is a homomorphism and $\varphi\circ\psi=1_{G'}$,
$\psi\circ\varphi=1_{G}$, i.e. $\psi$ is an inverse homomorphism to $\varphi$.
We say $G$ is isomorphic to $G'$ by $G\cong G'$ or
$\phi:G\xrightarrow{\sim}G'$.

\begin{bdefi}
  Let $(G,\star)$ be a group. A subset $H\subseteq G$ is a \define{subgroup}
  if $(H,\ast)$ is also a group.
\end{bdefi}

If $H\ne\emptyset$ and $H\subseteq G$, $H\le G$ or $H$ is a subgroup of $G$ if
and only if
\begin{enumerate}
  \item $H$ is closed under $\star$ ($\forall h_1,h_2\in H$,\ $h_1\star h_2\in
    H$).
  \item $H$ is closed under inverses ($h\in H\Rightarrow h^{-1}\in H$).
\end{enumerate}

\begin{note}
  The following is notation for arbitrary and abelian groups.
  \begin{center}
    $x\star y\rightarrow xy$ for arbitrary $G$, $x+y$ for abelian $G$\\
    $e\rightarrow1$ for arbitrary $G$, 0 for abelian $G$
  \end{center}
\end{note}

For an arbitrary subset $A\subseteq G$, and $g\in G$,
$$gA=\{ga:a\in A\}\hspace{50pt}Ag=\{ag:a\in
A\}\hspace{50pt}gAg^{-1}=\{gag^{-1}:a\in A\}$$

\begin{bthm}[Subgroup Criterion]
  Let $\emptyset\ne H\subseteq G$, $H\le G$ if and only if $\forall x,y\in H$,
  $xy^{-1}\in H$.
\end{bthm}

\begin{defi}
  Let $A\subseteq G$ be any subset. The \define{centralizer} of $A$ in $G$ is
  $C_G(A)=\{g\in G:gag^{-1}=a\}$ and it is the set of elements in $G$ which
  commute with all elements of $A$.
\end{defi}

\begin{prop}
  $C_G(A)\le G$
\end{prop}

\begin{pf}
  First we show that the centralizer is not empty. $1a=a1=a$, $\forall a\in A$
  $\Rightarrow$ $1\in C_G(A)$ $\Rightarrow$ $C_G(A)\ne0$ so the centralizer of
  $A$ is not empty. Let $x,y\in C_G(A)$. We want to show that $xy^{-1}\in
  C_G(A)$ or that $xy^{-1}\in C_G(A)$. We do this by showing that
  $\pn{xy^{-1}}a\pn{xy^{-1}}^{-1}=a$.
  \begin{align*}
    \pn{xy^{-1}}a\pn{xy^{-1}}^{-1}&=xy^{-1}ayx^{-1}\\
                                  &=x\pn{y^{-1}ay}x^{-1}\\
                                  &=xax^{-1}\tag*{($y\in C_G(A)$)}\\
                                  &=a\tag*{($x\in C_G(A)$)}
  \end{align*}
  Since this subset satisfies the Subgroup Criterion, the centralizer $C_G(A)$
  is a subgroup of $G$.
\end{pf}

\begin{defi}
  The \define{center} of a group $G$ is denoted $Z(G)=\{g\in G:gx=xg,\
  \forall x\in G\}$. $Z(G)=C_G(G)\le G$. $Z(G)$ is the set of elements of $G$
  which commute with all elements in $G$. If $G$ is abelian, $Z(G)=G$.
\end{defi}

\begin{defi}
  The \define{normalizer} of $A$ in $G$ is $N_G(A)=\{g\in G:gAg^{-1}=A\}$ or
  $\{g\in G:gag^{-1}=a'\in A\}$.
\end{defi}

\begin{prop}
  $C_G(A)\le N_G(A)\le G$
\end{prop}

\begin{defi}
  A \define{group action} of a group $G$ on a set $A$ is a map $G\times
  A\rightarrow A$ such that $(g_1g_2)\cdot a=g_1\cdot\pn{g_2\cdot a}$,
  $\forall g_1,g_2\in G$, $\forall a\in A$ and $1\cdot a=a$, $\forall a\in
  A$. It is denoted $G\circlearrowleft A$.
\end{defi}

\begin{defi}
  Suppose $G\circlearrowleft A$, the stabilizer of $a\in A$ in $G$ is
  $G_a=\{g\in G:g\cdot a=a\}$. $G_a\le G$.
\end{defi}

\begin{bdefi}
  An \define{equivalence relation} $\mathcal{E}$ on a set $S$ is a subset
  $\mathcal{E}\subseteq S\times S$ which is reflexive, symmetric, and
  transitive. We write $(a,b)\in\mathcal{E}\Leftrightarrow
  a\mathrel\mathcal{E}b$ or $a\sim b$.
  \begin{enumerate}
    \item $a\sim a$
    \item $a\sim b\Leftrightarrow b\sim a$
    \item $a\sim b$, $b\sim c\Rightarrow a\sim c$
  \end{enumerate}
\end{bdefi}

\begin{defi}
  The \define{equivalence class} of $a\in S$ is $[a]=\{b\in S:a\sim b\}$
\end{defi}

\begin{defi}
  The \define{quotient set} of $S$ under $\sim$ is $S$/$\sim=\{[a]:a\in S\}$.
\end{defi}

\begin{ex}
  $\mathbb{Q}=\{(a,b)\in\mathbb{Z}\times\mathbb{Z}:b\ne 0\}$/$\sim$,
  $(a,b)\sim(c,d)\Rightarrow ad=bc$.
\end{ex}

\begin{defi}
  The quotient set comes equipped with the \define{projection map}
  $\pi:S\rightarrow S$/$\sim$ where $a\mapsto[a]=\pi(a)$. This map is
  surjective by definition.
\end{defi}

\begin{bdefi}
  A group $G'$ is a \define{quotient group} of a group $G$ if
  \begin{enumerate}
    \item $G'=G$/$\sim$, $G'$ is the quotient set of $G$ under an equivalence
      relation $\sim$.
    \item The projection map $\pi:G\to G'=G$/$\sim$ is a group homomorphism.
  \end{enumerate}
\end{bdefi}

\begin{defi}
  Let $\varphi:G\to G'$ be a homomorphism and let $g'\in G'$. The
  \define{fiber} over $g'$ is $\varphi^{-1}(g')=\{g\in G:\varphi(g)=g'\}$.
\end{defi}

\begin{bprop}
  All quotient groups come from subgroups.
\end{bprop}

\begin{bpf}
  Let $\varphi:G\to G'$ be a homomorphism, then $\varphi$ induces an
  equivalence relation on $G$. Let $x\sim
  y\Leftrightarrow\varphi(x)=\varphi(y)$. But $\varphi$ is a group
  homomorphism, so
  $\varphi(x)=\varphi(y)\Leftrightarrow\varphi(x)\varphi(y)^{-1}
  =1_{G'}\Leftrightarrow\varphi(x)\varphi(y^{-1})=1\Leftrightarrow
  \varphi(xy^{-1})=1$. So $x\sim y\Leftrightarrow\varphi(xy^{-1})=1$. Let
  $K=\{g\in G:\varphi(g)=1\}$. Then $x\sim y\Leftrightarrow xy^{-1}\in K$.
  Recall $K=\ker\varphi\le G$.\\\\
  Let $G'$ be a quotient group of $G$. Then $x\sim
  y\Leftrightarrow[x]=[y]\Leftrightarrow\pi(x)=\pi(y)$ where $\pi:G\to G'$ is
  the projection. But $\pi(x)=\pi(y)\Leftrightarrow xy^{-1}\in\ker\varphi$.
\end{bpf}

\begin{defi}
  The \define{right coset} of a subgroup $H$ of a group $G$ by the element
  $x\in G$ is $Hx=\{hx:h\in H\}$. The \define{left coset}, denoted $xH$ is
  denoted similarly.
\end{defi}

\begin{prop}
  Let $\varphi:G\to G'$ be a homomorphism and $K=\ker\varphi$. Then
  $xKx^{-1}\subseteq K$, $\forall x\in G$.
\end{prop}

\begin{pf}
  We must show $\varphi(xkx^{-1})=1_{G'}$ for $x\in G$, $k\in K$. Then,
  $\varphi(xkx^{-1})=\varphi(x)\varphi(k)\varphi(x^{-1})
  =\varphi(x)\varphi(x)^{-1}=1_{G'}$.
\end{pf}

\begin{bdefi}
  The subgroup $N\le G$ is \define{normal} if $xNx^{-1}\subseteq N$ for all
  $x\in G$. It is denoted $N\unlhd G$.
\end{bdefi}

\begin{prop}
  $\ker\varphi\unlhd G$ for any homomorphism $\varphi:G\to G'$.
\end{prop}

\begin{bthm}
  Let $N\le G$. Then the following are equivalent.
  \begin{enumerate}
    \item $N\unlhd G$ ($xNx^{-1}\subseteq N,\ \forall x\in G$)
    \item $xNx^{-1}=N$
    \item $xN=Nx$
    \item $\forall x,y\in G,\ xy^{-1}\in N\Leftrightarrow y^{-1}x\in N$
  \end{enumerate}
\end{bthm}

\begin{bpf}
  $(1)\Rightarrow(2)$ Assume $\forall x\in G$, $xNx^{-1}\subseteq N$. We want
  to show $xNx^{-1}=N$. We do this by showing $N\subseteq xNx^{-1}$. Let $x\in
  G$, $n_0\in N$. We show $n_0\in xNx^{-1}$. Note that $x\in G\Rightarrow
  x^{-1}\in G$. Thus, $x^{-1}N\pn{x^{-1}}^{-1}\subseteq N$ since $N\unlhd G$.
  Thus there exists $n$ such that $x^{-1}nx=n_1\in N$.
  $n_0=x\pn{x^{-1}n_0x}x^{-1}=xn_1x^{-1}\in xNx^{-1}$.\\\\
  $(3)\Rightarrow(4)$ Assume $\forall x\in G$, $xN=Nx$. Let $x,y\in G$. We want
  to show $xy^{-1}\in N\Leftrightarrow y^{-1}x\in N$. So we must show this is
  true in both directions. Suppose $xy^{-1}\in N$. Then there exists an $n_1\in
  N$ such that $xy^{-1}=n_1$. Thus, $x=n_1y\in Ny=yN$ by assumption. So $x\in
  yN$. Thus there exists $n_2\in N$ such that $x=yn_2\Rightarrow y^{-1}x=n_2\in
  N$. Thus, $xy^{-1}\in N\Rightarrow y^{-1}x\in N$. Similarly, $y^{-1}x\in
  N\Rightarrow xy^{-1}\in N$.
\end{bpf}

\begin{prop}
  Let $H\le G$. Then, $x\sim y\Leftrightarrow y^{-1}x\in H$ is an equivalence
  relation on $G$.
\end{prop}

\begin{pf}
  We want to show $\sim$ is reflexive, symmetric, and transitive.
  \begin{enumerate}
    \item $x\sim x$: $x^{-1}x=1\in H$
    \item $x\sim y\Rightarrow y\sim x$: $x\sim y\Leftrightarrow y^{-1}x\in
      H\Rightarrow x^{-1}y\in H\Leftrightarrow y\sim x$
    \item $x\sim y$, $y\sim z\Rightarrow x\sim z$: $y^{-1}x\in H$, $z^{-1}y\in
      H\Rightarrow(z^{-1}y)(y^{-1}x)=z^{-1}x\in H\Leftrightarrow x\sim z$
  \end{enumerate}
  Thus, $\sim$ is an equivalence relation on $G$.
\end{pf}

Any subgroup gives an equivalence relation.

\begin{defi}
  An equivalence relation on a set $S$ is the same as a \define{partition} of
  $S$. $P=\{A_1,A_2,\ldots\}$, $A_i\subseteq S$ such that
  $S\cup_{i\in\mathbb{N}}A_i$, $A_i\cap A_j=\emptyset$, $i\ne j$. $a\sim
  b\Leftrightarrow a,b\in A_i$.
\end{defi}

\begin{prop}
  For $H\le G$, $x\sim y\Leftrightarrow y^{-1}x\in H\Leftrightarrow xH=yH\
  (Hx=Hy)$.
\end{prop}

\begin{pf}
  Suppose $y^{-1}x\in H$. We want to show that $xH=yH$ or $xH\subseteq yH$ and
  $yH\subseteq xH$. $y^{-1}x\in H$ implies that there exists a $h_1\in H$ such
  that $y^{-1}x=h_1$. Thus, $x=yh_1\Rightarrow x\in yH$. $y^{-1}x\in
  H\Leftrightarrow x^{-1}y\in H$ which implies that there exists a $h_2\in H$
  such that $x^{-1}y=h_2\Rightarrow y=xh_2\in xH$.
\end{pf}

\begin{note}
  $[x]=xH$.
\end{note}

\begin{prop}
  For $N\le G$, let $G/N=\{xN:x\in G\}$. Define $xN\cdot yN=(xy)N$. Then $G/N$
  is a group if and only if $N\unlhd G$.
\end{prop}

$G/N=G$/$\sim$ ($x\sim y\Leftrightarrow xN=yN$)\\

Every quotient group is $G/N$ for some $N$.\\

$\pi:G\rightarrow G$/$\sim$, $\ker\pi\unlhd G$, $G$/$\sim\ =G/\ker\pi$.

\begin{prop}
  If $H\le G$ and $G$ is abelian, then $H\unlhd G$.
\end{prop}

If $G$ is a group and $\sim$ is an equivalence relation on $G$, then the
quotient set $G$/$\sim$ is a quotient group if and only if the projection map
$\pi:G\rightarrow G$/$\sim$, $\pi(x)=[x]$ is a homomorphism.\\

If $N\unlhd G$, then $G/N$ is a quotient group, where $G/N=\{xN:x\in G\}$ and
$xN\cdot yN=(xy)N$.\\

These notions of quotient groups are equivalent.

\begin{prop}
  If $\sim$ is an equivalence relation and $G$/$\sim$ is a quotient group, then
  there exists a homomorphism $\pi:G\rightarrow G$/$\sim$ and $\ker\pi\unlhd
  G$.
\end{prop}

\begin{pf}
  $x\sim y\Leftrightarrow \pi(x)=\pi(y)\Leftrightarrow
  \pi(y^{-1}x)=1\Leftrightarrow y^{-1}x\in\ker\pi\Leftrightarrow
  x\ker\pi=y\ker\pi$.
\end{pf}

If $N\unlhd G$, define $x\sim y\Leftrightarrow xN=yN\Leftrightarrow y^{-1}x\in
N$. Then, $G$/$\sim=G/N$, $[x]=xN$, $\pi:G\to G/N$, $\pi(x)=xN$, $\ker\pi=N$.

\begin{prop}
  Every subgroup of an abelian group is a normal subgroup.
\end{prop}

\begin{defi}
  $S^n\subseteq\mathbb{R}^{n+1}$, $S^n=\{(x_1,x_2,\ldots,x_{n+1}):\sum
  x_i^2=1\}$
\end{defi}

For $H\le G$, the relation $x\sim y\Leftrightarrow xH=yH\Leftrightarrow
y^{-1}x\in H$ is an equivalence relation and thus partitions $G$ into
equivalence classes.

$$G=\bigcup\limits_{x\in G}[x],\ [x]\cap[y]=\emptyset,\ [x]\ne[y]$$
$$G=\bigcup\limits_{x\in G}xH,\ xH\cap yH=\emptyset,\ x\nsim y$$

\begin{prop}
  Let $H\le G$. The number of right cosets of $H$ equals the number of left
  cosets of $H$.
\end{prop}

\begin{pf}
  Let $R=\{Hx:x\in G\}$ and $L=\{xH:x\in G\}$. We construct a bijection
  $L\rightarrow R$. Define $f:R\to L$ by $f(Hx)=x^{-1}H$, and define $g:L\to R$
  by $g(xH)=Hx^{-1}$. Then $f$ and $g$ are mutually inverse. Hence
  $R\leftrightarrow L$.
\end{pf}

\begin{defi}
  The number of distinct left cosets of $H$ in $G$ is called the \define{index}
  of $H$ in $G$, and is denoted $[G:H]$.
\end{defi}

\begin{bthm}[Lagrange's Theorem]
  If $H$ is a subgroup of $G$, $|G|=|H|[G:H]$.
\end{bthm}

\begin{cor}
  In a finite group, the order of every element divides the order of the group.
\end{cor}

\begin{cor}
  A group of prime order is cyclic.
\end{cor}

\begin{cor}
  Let $G$ be a finite group and let $a\in G$. Then, $a^{|G|}=1$.
\end{cor}

Let $\varphi:G\to G'$ be a homomorphism. How far is $\varphi$ from an
isomorphism? How can $\varphi$ fail to be an isomorphism?
\begin{enumerate}
  \item $\varphi$ could fail to be injective. $(\ker\varphi\ne\{1\})$
  \item $\varphi$ could fail to be surjective.
\end{enumerate}

\begin{bthm}[First Isomorphism Theorem]
  Let $\varphi:G\to G'$ be a homomorphism. Then $\ker\varphi\unlhd G$,
  $\im\varphi\le G'$ and
  $$G/\ker\varphi\cong\im\varphi$$
\end{bthm}

\begin{prop}
  There exists an isomorphism $\theta:G/\ker\varphi\to \im\varphi$ such that
  \begin{center}
    \begin{tikzcd}[column sep={8em,between origins}, row sep=6em, every
      matrix/.append style={name=m}, execute at end picture={
        \draw [<-] ([xshift=-4em, yshift=2.25em]m-2-2.north)
        arc[start angle=-90,delta angle=270,radius=0.25cm];
      }]
      G \arrow[r, "\varphi"] \arrow[d, "\pi"] & G' \arrow[d,
      hookleftarrow, "\iota"] \\
      G/\ker\varphi \arrow[r, "\theta"] & \im\varphi
    \end{tikzcd}
  \end{center}
  The curved arrow in the middle means the diagram is commutative, i.e.
  $\varphi=\iota\cdot\theta\cdot\pi$. The curved arrow means it is injective.
\end{prop}

\begin{pf}
  Define $\theta:G/\ker\varphi\to \im\varphi$ by
  $\theta(x\ker\varphi)=\varphi(x)$.\\
  First we show that $\theta$ is well-defined. Suppose
  $x\ker\varphi=y\ker\varphi$. Then,
  \begin{align*}
    x\ker\varphi=y\ker\varphi
    &\Leftrightarrow y^{-1}x\ker\varphi=\ker\varphi\\
    &\Leftrightarrow y^{-1}x\in\ker\varphi\\
    &\Leftrightarrow \varphi(y^{-1}x)=1\\
    &\Leftrightarrow \varphi(y)^{-1}\varphi(x)=1\\
    &\Leftrightarrow \varphi(x)=\varphi(y)\\
    &\Leftrightarrow \theta(x\ker\varphi)=\theta(y\ker\varphi)
  \end{align*}
  Thus, $\theta$ is well-defined.\\\\
  Then, we show that $\theta$ is a homomorphism. Let $K=\ker\varphi$.
  \begin{align*}
    \theta(xKyK)&=\theta(xyK)\\
                &=\varphi(xy)\\
                &=\varphi(x)\varphi(y)\\
                &=\theta(xK)\theta(yK)
  \end{align*}
  Thus, $\theta$ is a homomorphism.\\\\
  Then, we show that $\theta$ is injective.
  \begin{align*}
    \theta(xK)=\theta(yK)&\Leftrightarrow\varphi(x)=\varphi(y)\\
                         &\Leftrightarrow\varphi(y)^{-1}\varphi(x)=1\\
                         &\Leftrightarrow\varphi(y^{-1}x)=1\\
                         &\Leftrightarrow y^{-1}x\in K\\
                         &\Leftrightarrow xK=yK
  \end{align*}
  Thus, $\theta$ is injective.\\\\
  Then, we show that $\theta$ is surjective. Let $y\in\im\varphi$. There exists
  $xK\in G/K$ such that $\theta(xK)=y$. We know there exists an $x\in G$ such
  that $\varphi(x)=y$. $\theta(xK)=\varphi(x)=y$. Thus, $\theta$ is surjective
  and $\theta$ is an isomorphism.
\end{pf}

\begin{prop}
  Let $a\in G$. If $|a|=\infty$, then $\langle a\rangle\cong(\mathbb{Z},+)$.
  If $|a|=n$, then $\langle a\rangle=\mathbb{Z}_n=\mathbb{Z}/n\mathbb{Z}$.
\end{prop}

\begin{pf}
  Consider $\mathbb{Z}\xrightarrow{\pi} G$ defined by $\pi(k)=a^k$.
\end{pf}

\begin{defi}
  Let $(A,\star)$ and $(B,\ast)$ be groups. The \define{direct product} or
  \define{direct sum} of $A$ and $B$ is $A\oplus B=\{(a,b):a\in A,b\in B\}$
  where $(a_1,b_1)\cdot(a_2,b_2)=(a_1\star a_2,b_1\ast b_2)\in A\oplus B$.
\end{defi}

\begin{defi}
  In a group $G$, define $a\sim b\Leftrightarrow\exists x\in G$ such that
  $b=xax^{-1}$. This is an equivalence relation and $a$ and $b$ are
  \define{conjugates}.
\end{defi}

\begin{defi}
  For any $x\in G$, the \define{inner automorphism} of $G$ induced by $x$ is
  $T_x:G\to G$ defined by $T_x(g)=xgx^{-1}$.
\end{defi}

\begin{defi}
  The set of all inner automorphisms of $G$ is a group, called the
  \define{inner automorphism group}, and is denoted $\Inn(G)=\{T_x:G\to G\mid
  x\in G\}$.
\end{defi}

\begin{prop}
  $G/Z(G)\cong\Inn(G)$
\end{prop}

\begin{pf}
  Consider $\psi:G\to \Inn(G)$ defined by $x\mapsto T_x$. Then, $\psi$ is
  surjective, i.e. $\im\psi=\Inn(G)$. We then determine the kernel of the
  homomorphism.
  \begin{align*}
    \ker\psi&=\{x\in G:\psi(x)=1_G\}\\
            &=\{x\in G:T_x(g)=g,\ \forall g\in G\}\\
            &=\{x\in G:xgx^{-1}=g,\ \forall g\in G\}\\
            &=\{x\in G:xg=gx,\ \forall g\in G\}\\
            &=Z(G)
  \end{align*}
  By the first isomorphism theorem, $G/Z(G)\cong\Inn(G)$.
\end{pf}

\begin{bthm}[Third Isomorphism Theorem]
  Let $G$ be a group. Let $A\unlhd G$, $B\unlhd G$. If $A\subseteq B$, then
  $A\unlhd B$, $B/A\unlhd G/A$, and $$(G/A)/(B/A)\cong(G/B)$$
\end{bthm}

\begin{bpf}
  First we establish $A\unlhd B$. $A\le B$ because $A\le G$ and $A\subseteq B$.
  $$A\unlhd B\Leftrightarrow bAb^{-1}\subseteq A,\ \forall b\in B$$
  $$A\unlhd G\Leftrightarrow xAx^{-1}\subseteq A,\ \forall x\in G$$
  But $B\subseteq G$ so $b\in G$. Thus, $bAb^{-1}\subseteq A$, $\forall b\in B$
  and $A\unlhd B$. Thus, $A\unlhd B$ and we may construct $B/A$.\\\\
  We first show $B/A\le G/A$. It is closed under multiplication since
  $(b_1A)(b_2A)=(b_1b_2)A\in B/A$ because $B$ is a group. It is also closed
  under inverses since $(bA)^{-1}=b^{-1}A\in B/A$.\\\\
  We then show $B/A\unlhd G/A$ by showing $x(B/A)x^{-1}\subseteq B/A$, $\forall
  x\in G/A$. Let $x\in G/A\Leftrightarrow yA$, $y\in G$. We want to show
  $(yA)(B/A)(yA)^{-1}\subseteq B/A$. Let $z\in(yA)(B/A)(yA)^{-1}$. Then, there
  exist $a_1,a_2\in A$, $b_1\in B$ such that
  \begin{align*}
    z&=(ya_1)(b_1A)(y^{-1}a_2)\\
     &=y(a_1b_1)Ay^{-1}a_2\\
     &=y(a_1b_1)y^{-1}Aa_2
  \end{align*}
  We know $a_2\in A\Rightarrow Aa_2=A$ and $A\subseteq B\Rightarrow a_1\in
  A\subseteq B\Rightarrow a_1\in A\Rightarrow a_1b_1\in B$. Thus, there exists
  $b_2\in B$ such that $a_1b_1=b_2$. We substitute these in to get
  $$z=yb_2y^{-1}A$$
  We know $B\unlhd G\Rightarrow yBy^{-1}\subseteq B$. Thus, there exists a
  $b_3\in B$ such that $yb_2y^{-1}=b_3\in B$. We then get $z=b_3A$. Since
  $z=b_3A\in B/A$, $B/A\unlhd G/A$.\\\\
  Now we prove $(G/A)/(B/A)\cong(G/B)$. We define the homomorphism
  $\omega:G/A\to G/B$ such that $\omega(xA)=xB$. We show that $\omega$ is
  well-defined. If $xA=yA$, then
  \begin{align*}
    xA=yA&\Leftrightarrow y^{-1}x\in A\subseteq B\\
         &\Rightarrow y^{-1}x\in B\\
         &\Leftrightarrow xB=yB\\
         &\Leftrightarrow\omega(xA)=\omega(yA)
  \end{align*}
  We may then determine the kernel and image of the homomorphism.
  $$\im\omega=\{xB:x\in G\}=G/B$$
  $$\ker\omega=\{xA:\omega(xA)=B\}=\{xA:xB=B\}=\{xA:x\in B\}=B/A$$
  By the first isomorphism theorem, $(G/A)/\ker\omega\cong\im\omega$ so
  $(G/A)/(B/A)\cong(G/B)$.
\end{bpf}

\begin{prop}
  There is an isomorphism $\theta:(G/A)/(B/A)\to G/B$ such that this diagram
  commutes.
  \begin{center}
    \begin{tikzcd}[column sep={6em,between origins}, row sep=huge]
                           & G/A \arrow[dl, leftarrow, "\pi"] \arrow[dr,
      "\sigma"] \arrow[dd, "\omega"]& \\
      G \arrow[dr, "\rho"] & & (G/A)/(B/A) \arrow[dl, "\theta"] \\
                           & G/B &
    \end{tikzcd}
  \end{center}
\end{prop}

\begin{bthm}[Second Isomorphism Theorem]
  Let $G$ be a group, $A\le G$, and $N\unlhd G$. Then $AN\le G$, $N\unlhd AN$,
  and $A\cap N\unlhd A$. Also, $$(AN)/N\cong A/(A\cap N)$$
\end{bthm}

\begin{bpf}
  Let $\varphi:A\to AN/N$ such that $a\mapsto aN$. Then by the first
  isomorphism theorem, $(AN)/N\cong A/(A\cap N)$.
\end{bpf}

\begin{ex}
  We look at an example of the third isomorphism theorem. Let $G=\mathbb{Z}$,
  $A=12\mathbb{Z}$, and $B=4\mathbb{Z}$. We observe that $A\unlhd B\unlhd G$ so
  the conditions for the third isomorphism theorem are satisfied.
  $$G/A=\mathbb{Z}/12\mathbb{Z}=\{0,1,\ldots,11\}(\text{mod }12)$$
  $$B/A=4\mathbb{Z}/12\mathbb{Z}=\{0,4,8\}(\text{mod }12)$$
  $$(G/A)/(B/A)=\{0,1,2,3\}(\text{mod }4)=\mathbb{Z}/4\mathbb{Z}$$
  $$(\mathbb{Z}/12\mathbb{Z})/(4\mathbb{Z}/12\mathbb{Z})
  \cong\mathbb{Z}/4\mathbb{Z}$$
\end{ex}

\begin{ex}
  We look at an example of the second isomorphism theorem. Let $G=\mathbb{Z}$,
  $N=12\mathbb{Z}$, and $A=8\mathbb{Z}$.
  $$A\cap N=\{0,\pm24,\pm48,\ldots\}=24\mathbb{Z}$$
  $$AN=\{0,\pm4,\pm8,\ldots\}=4\mathbb{Z}$$
  $$AN/A=4\mathbb{Z}/12\mathbb{Z}=\{0,4,8\}(\text{mod }12)$$
  $$A/(A\cap N)=8\mathbb{Z}/24\mathbb{Z}
  =\{0,8,16\}(\text{mod }24)$$
  $$AN/N\cong\mathbb{Z}/3\mathbb{Z}\cong A/(A\cap N)$$
\end{ex}

\begin{bdefi}
  A ring $(R,+,\cdot)$ is a set together with two binary operations, called
  addition and multiplication respectively, satisfying the following three
  axioms.
  \begin{enumerate}[(a)]
    \item The set $(R,+)$ together with addition is an abelian group.
    \item The binary operation $\cdot$ is associative on $R$.
    \item The distributive law holds in $R$; for all $a,b,c\in R$,
      $$(a+b)\cdot c=(a\cdot c)+(b\cdot c)$$
      $$a\cdot(b+c)=(a\cdot b)+(a\cdot c)$$
  \end{enumerate}
\end{bdefi}

\begin{defi}
  The ring $R$ is \define{commutative} if multiplication is commutative.
\end{defi}

\begin{defi}
  The ring $R$ has an \define{identity}, or \define{unity} or contains a $1$ if
  there is an element $1\in R$ such that for all $a\in R$, $1\cdot
  a=a\cdot1=a$.
\end{defi}

\begin{note}
  By abuse of notation, multiplication $\cdot$ may be denoted by simple
  juxtaposition, i.e. $a\cdot b=ab$.
\end{note}

\begin{note}
  For a ring with 1, the condition of commutativity under addition is
  redundant. Note that for any $a,b\in R$,
  $$(1+1)(a+b)=1(a+b)+1(a+b)=a+b+a+b$$
  $$(1+1)(a+b)=(1+1)a+(1+1)b=a+a+b+b$$
  Therefore, $a+b+a+b=a+a+b+b$ and therefore $a+b=b+a$. Thus $R$ is abelian.
\end{note}

\begin{defi}
  A ring with identity is a \define{division ring} if every nonzero element has
  a multiplicative inverse.
\end{defi}

\begin{bdefi}
  A \define{field} is a commutative division ring.
\end{bdefi}

\begin{ex}
  The following are two very simple examples of rings.\\\\
  \textbf{The Zero Ring}\\
  Let $R=\{0\}$. Then $R$ is a ring and is called the zero ring.\\
  \textbf{Trivial Rings}\\
  For any abelian group $(G,+)$, consider the ring $(G,+,\cdot)$, where
  multiplication is given by $a\cdot b=0$ for any $a,b\in G$.
\end{ex}

\begin{prop}
  Let $R$ be a ring, and $a,b\in R$.
  \begin{enumerate}[(a)]
    \item $0a=a0=0$
    \item $(-a)b=a(-b)=-(ab)$, where $-(a)$ is the additive inverse of $a$.
    \item $(-a)(-b)=ab$
    \item If $R$ has identity 1, then it is unique and $-a=(-1)a$.
  \end{enumerate}
\end{prop}

\begin{defi}
  A nonzero element element $a$ of a ring $R$ is a \define{zero divisor} if
  there is a nonzero $0\ne b\in R$ such that $ab=0$ or $ba=0$.
\end{defi}

\begin{defi}
  Let $R$ be a ring with identity. An element $a$ of $R$ is a \define{unit} if
  it has a multiplicative inverse, i.e. there is some $b\in R$ such that
  $ab=ba=1$. The set of units of $R$ is denoted $R^\times$.
\end{defi}

\begin{bdefi}
  An \define{integral domain} is a commutative ring with identity which has no
  zero divisors.
\end{bdefi}

\begin{prop}
  Let $R$ be an integral domain, and let $a,b,c\in R$. If $ab=ac$, then $a=0$
  or $b=c$.
\end{prop}

\begin{defi}
  Let $R$ be a commutative ring with 1. For any $a_0,a_1,\ldots,a_n\in R$, the
  expression
  $$p(x)=a_0+a_1x+a_2x^2+\ldots+a_nx^n$$
  is a \define{polynomial} in $R$ with coefficients $a_0,a_1,\ldots,a_n$. If
  $a_n\ne0$, then $p(x)$ has \define{degree} $n$. The set of all polynomials in
  $R$ is denoted $R[x]$ or $R$ adjoin $x$. $R[x]$ is a ring (called the ring of
  polynomials in $R$ in one variable) under "usual" addition and
  multiplication. Let $p(x)=a_0+a_1x+\ldots+a_nx^n$ and
  $q(x)=b_0+b_1x+\ldots+b_mx^m$, and without loss of generality $n>m$. Then,
  $$p(x)+q(x)=(a_0+b_0)+(a_1+b_1)x+\ldots+(a_n+b_n)x^n$$
  where $b_k=0$ for $k>m$ and
  $$p(x)q(x)=\sum_{k=0}^{m+n}(\sum_{i+j=k}a_ib_j)x^k$$
\end{defi}

\begin{note}
  Polynomials are not determined by their values
\end{note}

The following is a formal construction of the ring of polynomials in $R$.\\

Let $R$ be a commutative ring with 1. $R[x]$ is the set of all tuples
$p(x)=(a_0,a_1,\ldots,a_n)\in
R^\infty=\prod_{i\in\mathbb{N}}R=\oplus_{i\in\mathbb{N}}R$, i.e. $a_k\in R$
where $\exists n\in\mathbb{N}$ such that $a_k=0$ for $k>n$. The smallest such
$n$ is the degree of $p(x)$. If $p=(a_0,a_1,\ldots,a_n,0,\ldots)$ and
$q=(b_0,b_1,\ldots,b_m,0,\ldots)$, then
$$p+q=(a_0+b_0,a_1+b_1,\ldots,a_n+b_n,0,\ldots)$$
$$pq=(c_0,c_1,\ldots,c_k,0,\ldots),\ c_k=\sum_{i+j=k}a_ib_j$$

\begin{defi}
  Let $R$ be any ring $M_n(R)=\{n\times n\text{ matrices with entries in
  $R$}\}$, $A=(a_{ij})$, $B=(b_{ij})$, $(A+B)_{ij}=a_{ij}+b_{ij}$, $A\cdot
  B=C$, $c_{ij}=\sum_{k=1}^na_{ik}b_{kj}$. This is the ring of \define{$n\times
  n$ matrices over $R$} or with entries in $R$. If $R$ has 1, then
  $$I=\begin{bmatrix}
    1 & \ldots & 0 \\
    \vdots & \ddots & \vdots \\
    0 & \ldots & 1
  \end{bmatrix}=1\in M_n(R)$$
\end{defi}

\begin{defi}
  $GL_n(R)$ is the group of units of $M_n(R)$ and is called the \define{general
  linear group}.
\end{defi}

\begin{defi}
  Let $R$ be commutative with 1. Let $G=\{g_1,\ldots,g_n\}$ be a finite group.
  The \define{group ring} $RG$ of $G$ with coefficients in $R$ is the set of
  all formal sums
  $$a_1g_1+a_2g_2+\ldots+a_ng_n$$
  where $a_i\in R$,
  $$(a_1g_1+\ldots+a_ng_n)+(b_1g_1+\ldots+b_ng_n)
  =(a_1+b_1)g_1+\ldots+(a_n+b_n)g_n$$
  $$(a_1g_1+\ldots+a_ng_n)\cdot(b_1g_1+\ldots+b_ng_n)
  =c_1g_1+\ldots+c_ng_n,\text{ where $c_k=\sum_{g_ig_j=g_k}a_ib_j$}$$
\end{defi}

\begin{note}
  $1\cdot g_i=g_i$, $a_i\cdot1=a_i$, $(a_ig_i)(b_jg_j)=(a_ib_j)(g_ig_j)$
\end{note}

\begin{ex}
  $G=S_4$, $R=\mathbb{Z}$.
  $$x=2(1\ 2)+(2\ 3)+7(1\ 2\ 4)\hspace{50pt} y=3(1)+2(2\ 3)$$
  \begin{align*}
    x+y&=3(1)+2(1\ 2)+3(2\ 3)+7(1\ 2\ 4)\\
    xy&=6(1\ 2)+4(1\ 2)(2\ 3)+3(2\ 3)+2(1)+21(1\ 2\ 4)+14(1\ 2\ 4)(2\ 3)\\
      &=2(1)+6(1\ 2)+3(2\ 3)+4(1\ 2\ 3)+21(1\ 2\ 4)+14(1\ 2\ 3\ 4)
  \end{align*}
\end{ex}

\begin{bdefi}
  A \define{subring} $S$ of a ring $(R,+,\cdot)$ is a subgroup $S\le(R,+)$
  which is closed under the multiplicative structure of $R$.
\end{bdefi}

\begin{prop}
  A subset $S$ of the ring $R$ is a subring if and only if $S$ is closed under
  subtraction and multiplication.
\end{prop}

\begin{pf}
  This follows immediately from the fact that a subset $H$ of an abelian group
  $G$ is a subgroup if and only if $H$ is closed under subtraction.
\end{pf}

\begin{defi}
  The \define{center} of a ring $A$ is the set of elements $a\in A$ such that
  $ax=xa$ for all $x\in A$. The center of $A$ is a subring of $A$.
\end{defi}

\begin{bdefi}
  Let $R$ and $S$ be rings. A \define{ring homomorphism} is a map of sets
  $\varphi:R\to S$ such that for all $a,b\in R$,
  $$\varphi(a+b)=\varphi(a)+\varphi(b)$$
  $$\varphi(ab)=\varphi(a)\varphi(b)$$
\end{bdefi}

\begin{defi}
  The \define{kernel} of the homomorphism $\varphi:R\to S$ is given by
  $$\ker\varphi=\{r\in R:\varphi(r)=0\in S\}$$
\end{defi}

\begin{defi}
  A \define{ring isomorphism} is a bijective homomorphism.
\end{defi}

\begin{bdefi}
  A subring $I$ of $R$ is a \define{left ideal} of $R$ if $I$ is closed under
  left multiplication by elements from $R$, i.e. $rI\subseteq I$ for all $r\in
  R$. Similarly, $I$ is a \define{right ideal} of $R$ if $I$ is closed under
  right multiplication by elements of $R$, i.e. $Ir\subseteq I$ for all $r\in
  R$. A subring which is both a left and right ideal is called a \define{two
  sided ideal}, or simply \define{ideal}.
\end{bdefi}

\begin{bdefi}
    The \define{quotient ring} $R/I$ of the ring $R$ by the ideal $I\subseteq
    R$ is the quotient group of cosets $R/I$ under the operations
    $$(r+I)+(s+I)=(r+s)+I\hspace{50pt}(r+I)\cdot(s+I)=(r\cdot s)+I$$
    for all $r,s\in R$.
\end{bdefi}

\begin{prop}
  For any ring $R$ and ideal $I$, $R/I$ is a ring.
\end{prop}

\begin{prop}
  If $I$ is any ideal of $R$, the map $\varphi:R\to R/I$ defined by $r\mapsto
  r+I$ is a surjective ring homomorphism with kernel $I$.
\end{prop}

\begin{thm}[First Isomorphism Theorem for Rings]
  If $\varphi:R\to S$ is homomorphism of rings, then $\ker\varphi$ is an ideal
  of $R$, $\im\varphi$ is a subring of $S$, and
  $$R/\ker\varphi\cong\im\varphi$$
\end{thm}

\begin{thm}[Second Isomorphism Theorem for Rings]
  Let $R$ be a ring, $A$ a subring and $B$ an ideal of $R$. Then
  $A+B=\{a+b:a\in A,b\in B\}$ is a subring of $R$, $A\cap B$ is an ideal of $A$
  and
  $$(A+B)/B\cong A/(A\cap B)$$
\end{thm}

\begin{thm}[Third Isomorphism Theorem for Rings]
  Let $I$ and $J$ be ideals of the ring $R$ such that $I\subseteq J$. Then
  $J/I$ is an ideal of $R/I$ and
  $$(R/I)(J/I)\cong(R/J)$$
\end{thm}

\begin{thm}[Fourth Isomorphism Theorem for Rings]
  Let $I$ be an ideal of $R$. The correspondence
  $$A\longleftrightarrow A/I$$
  is an inclusion preserving bijection between the subring $A$ of $R$
  containing $I$ and the set of subrings of $R/I$. Further, a subring $A$
  containing $I$ is an ideal of $R$ if and only if $A/I$ is an ideal of $R/I$.
\end{thm}

\begin{bdefi}
  Let $A\subseteq R$ be a subset. Then the \define{ideal generated by $A$} is
  the smallest ideal of $R$ containing $A$, and is denoted $(A)$.
\end{bdefi}

\begin{note}
  In this context, "smallest" means all other ideals containing $A$ also
  contain $(A)$. In other words, $A\subseteq J\implies(A)\subseteq J$.
\end{note}

\begin{prop}
  $(A)$ is the intersection of all ideal $I$ containing $A$, or
  $$(A)=\bigcap_{A\subseteq I}I$$
\end{prop}

\begin{pf}
  $R$ is an ideal of itself containing $A$ and the intersection of nonempty
  ideals is an ideal. By definition the intersection contains $A$. Therefore,
  $\bigcap_{A\subseteq I}I$ is an ideal containing $A$. Since $(A)$ is the
  smallest ideal containing $A$, $(A)\subseteq\bigcap_{A\subseteq I}I$.\\\\
  On the other hand, suppose $x\in\bigcap_{A\subseteq I}I$. Then for any ideal
  $I$ containing $A$, $x\in I$. But $(A)$ is an ideal containing $A$. Thus
  $x\in(A)$. Therefore, $\bigcap_{A\subseteq I}I\subseteq(A)$. Thus,
  $(A)=\bigcap_{A\subseteq I} I$.
\end{pf}

\begin{prop}
  If $R$ is commutative, then
  $$(A)=RA=AR$$
  where
  $$RA=\{r_1a_1+r_2a_2+\ldots+r_na_n:r_i\in R, a_i\in A, n\in\mathbb{Z}\}$$
  and $AR$ is define similarly.
\end{prop}

\begin{defi}
  An ideal generated by a single element is called a \define{principal ideal}.
\end{defi}

\begin{defi}
  An ideal generated by a finite set is called a \define{finitely generated
  ideal}.
\end{defi}

\begin{defi}
  An ideal $I$ of a ring $R$ is \define{proper} if it is a proper subset, i.e.
  $I\ne R$ and $I\subsetneq R$.
\end{defi}

\begin{defi}
  A proper ideal $M$ of a ring $R$ is \define{maximal} if whenever $I$ is an
  ideal of $R$ and $M\subseteq I\subseteq R$, then $M=I$ or $M=R$.
\end{defi}

\begin{ex}
  Consider $(x-4)\in\mathbb{R}[x]$.
  $$(x-4)=(\{x-4\})=\{f(x)(x-4):f\in\mathbb{R}[x]\}$$
  We claim $(x-4)$ is maximal in $\mathbb{R}[x]$. Suppose $(x-4)\subsetneq
  I\subseteq R$. We want to show $I=R=\mathbb{R}[x]$. There exists $f(x)\in I$
  with $f(x)\notin(x-4)$. Recall polynomial long division. $\forall
  f(x),g(x)\in\mathbb{R}[x],\exists q(x),r(x)\in\mathbb{R}[x]$ such that
  $$f(x)=q(x)g(x)+r(x),\deg r(x)<\deg g(x)$$
  In our case, $g(x)=(x-4)$, with $\deg r(x)<1$. This implies that
  $r(x)=r\in\mathbb{R}$ so we can rewrite our expression as
  $$f(x)=q(x)(x-4)+r$$
  Since $(x-4)\subsetneq I$, we know $x-4\in I$ and $q(x)(x-4)\in I$. Since $I$
  is a subring, $f(x)-q(x)(x-4)=r\in I$. Since $f(x)\notin(x-4)$ and
  $q(x)(x-4)\in(x-4)$, $r\ne0$. Since $0\ne r\in\mathbb{R}$, $r$ is a unit in
  $\mathbb{R}[x]$. If $r\in I$ is a unit, then $I=R$ because $r$ being a unit
  $\implies u^{-1}\in R\implies u^{-1}u\in I\implies1\in I\implies\forall r\in
  R, r1\in I\implies I=R$. Therefore, $I=\mathbb{R}[x]$ and $(x-4)$ is maximal.
\end{ex}

\begin{defi}
  A proper ideal $P$ of a commutative ring $R$ is \define{prime} if $ab\in P$
  implies $a\in P$ or $b\in P$ for any $a,b\in R$.
\end{defi}

\begin{ex}
  Consider $2\mathbb{Z}\subseteq\mathbb{Z}$. We claim $2\mathbb{Z}$ is a prime
  ideal. Let $a,b\in\mathbb{Z}$. Suppose $ab\in2\mathbb{Z}$. Then $a$ or $b$ is
  even, i.e. $a\in2\mathbb{Z}$ or $b\in2\mathbb{Z}$. Therefore, $2\mathbb{Z}$
  is prime.\\\\
  \textbf{Alternate Proof:} $ab\in2\mathbb{Z}\Leftrightarrow\exists
  n\in\mathbb{Z}$ such that $ab=2n$. Using prime factorization, there exist
  primes $p_1,\ldots,p_l,q_1,\ldots,q_s$ such that $a=p_1\ldots p_l$ and
  $b=q_1\ldots q_s$. Thus, $p_1\ldots p_lq_1\ldots q_s=2n$ and there exists $i$
  or $j$ such that $p_i=2$ or $q_j=2$. Thus, $a\in2\mathbb{Z}$ or
  $b\in2\mathbb{Z}$ and $2\mathbb{Z}$ is prime.
\end{ex}

\begin{bthm}
  Let $R$ be a commutative ring with identity and let $A\subseteq R$ be an
  ideal. Then $R/A$ is an integral domain if and only if $A$ is prime.
\end{bthm}

\begin{bpf}
  Suppose $R/A$ is an integral domain. Let $a,b\in R$ and suppose that $ab\in
  A$. We must show $a\in A$ or $b\in A$. We compute $(a+A)(b+A)=ab+A=A=0+A$,
  which is the additive identity in $R/A$. But $R/A$ is an integral domain so
  $a+A=A$ or $b+A=A$. Therefore, $a\in A$ or $b\in A$.\\\\
  Conversely, supposed that $A$ is prime and let $a+A,b+A\in R/A$ such that
  $(a+A)(b+A)=ab+A=A$. Then $ab\in A$. But $A$ is prime so $a\in A$ or $b\in
  A$. Thus, $a+A=0\in R/A$ or $b+A=0\in R/A$ and $R/A$ is an integral domain.
\end{bpf}

\begin{bthm}
  Let $R$ be a commutative ring with identity and let $A$ be an ideal of $R$.
  Then $R/A$ is a field if and only if $A$ is maximal.
\end{bthm}

\begin{bpf}
  Suppose $R/A$ is a field. Let $B$ be an ideal of $R$ that properly contains
  $A$, $A\subsetneq B\subseteq R$. We want to show that $B=R$. There exists
  $b\in B$ such that $b\notin A$. Then $b+A$ is a nonzero element of $R/A$.
  But $R/A$ is a field, hence $b+A$ must have a multiplicative inverse, i.e.
  there exists $c\in R$ such that $(b+A)(c+A)=bc+A=1+A$. Therefore, $1-bc\in
  A\subsetneq B$. But $bc\in B$ since $B$ is an ideal so $(1-bc)+bc=1\in B$.
  Since $1\in B$, $B=R$.\\\\
  Conversely, suppose that $A$ is maximal. We want to show that $R/A$ is a
  field. Since $R$ is commutative and has an identity, $R/A$ is also
  commutative and has an identity. We want to show that every nonzero element
  of $R/A$ has a multiplicative inverse. Every nonzero element of $R/A$ is of
  the form $b+A$, $b\in R-A$. Choose and fix such an element $b$. Consider the
  subset $B\subseteq R$ such that
  $$B=\{br+a:r\in R,a\in A\}$$
  We want to show that $B$ is an ideal of $R$ properly containing $A$. Since
  $$(br+a)-(br'+a')=b(r-r')+(a-a')\in B$$
  we know that $B$ is a subgroup of $(R,+)$. We also know that is it closed
  under multiplication since
  $$(br+a)(br'+a')=brbr'+bra'+br'a+aa'=b(rbr'+ra'+r'a)+(aa')\in B$$
  so $B$ is a subring. Also for any $s\in R$,
  $$s(br+a)=sbr+sa=b(sr)+(sa)$$
  Because $A$ is an ideal, $sa\in A$ so $B$ is an ideal of $R$. Also for any
  $a\in A$, $a=b0+a\in B$ and $b=b1+0\in B-A$ so $B$ is an ideal that properly
  contains $A$. However, $A$ is maximal so $B=R$. Because $R$ contains 1, there
  exists $c\in R$ and $a'\in A$ such that $1=bc+a'$. If we consider the coset
  of $R/A$ this element is in, we see that $1+A=bc+a'+A$. Since $a'\in A$, we
  can rewrite our equation as $1+A=(b+A)(c+A)$. Therefore, for any $b+A\in
  R/A$, there exists a multiplicative inverse and $R/A$ is a field.
\end{bpf}

\begin{prop}
  In a commutative ring $R$ with identity, every maximal ideal is prime.
\end{prop}

\begin{defi}
  A \define{norm} $N$ on the integral domain $R$ is a map of set
  $N:R\to\mathbb{N}\cup\{0\}$. If $N(a)>0$, $\forall a\in R$, we say $N$ is a
  \define{positive norm}.
\end{defi}

\begin{note}
  Some texts require for nonzero $a,b\in R$, $N(a)\le N(ab)$. Also, it is not
  required that $N(a+b)\le N(a)+N(b)$ or $N(ab)\le N(a)N(b)$.
\end{note}

\begin{defi}
  The integral domain $R$ is called a \define{Euclidean domain} if there is a
  norm $N$ on $R$ such that if $a,b\in R$, $b\neq 0$, then $\exists q,r\in R$
  such that $a=qb+r$ where $r=0$ or $N(r)<N(b)$. Here, $q$ is the
  \define{quotient} and $r$ is the \define{remainder}.
\end{defi}

\begin{defi}
  The \define{Euclidean algorithm} for two elements $a,b$ in a Euclidean domain
  $R$ is a list of divisions
  \begin{align*}
    a&=q_0b+r_0\\
    b&=q_1r_0+r_1\\
    r_0&=q_2r_1+r_2\\
       &\ldots\\
    r_{n-1}&=q_{n+1}r_n
  \end{align*}
  where $r_n$ is the last nonzero remainder. Such an $r_n$ exists as
  $N(r_1)>N(r_2)>\ldots>N(r_n)\ge0$ is a decreasing sequence of nonnegative
  integers.
\end{defi}

\begin{ex}
  If $K$ is a field, then $K[x]$ is a Euclidean domain with norm
  $N(f(x))=\deg(f(x))$. The division algorithm is polynomial long divison. Let
  $f,g\in\mathbb{Z}_5[x]$, $f=3x^4+x^3+2x^2+1$, and $g=x^2+4x+2$. Then we have
  $3x^4+x^3+2x^2+1=(3x^2+4x)(x^2+4x+2)+(2x+1)$.
\end{ex}

\begin{ex}
  Gaussian integers $\mathbb{Z}[i]=\{a+bi:a,b\in \mathbb{Z}\}$, $ii=-1$, with
  norm $N(a+bi)=a^2+b^2$.
\end{ex}

\begin{defi}
  Let $a,b\in R$, $b\ne0$. Then $a$ is a \define{multiple} of $b$ if $a=qb$ for
  some $q\in R$. We also say $b$ \define{divides} $a$ or $b$ is a
  \define{divisor} of $a$, or $b\vert a$.
\end{defi}

\begin{defi}
  The \define{greatest common divisor} of $a$ and $b$ is a nonzero element
  $d\in R$ such that
  \begin{enumerate}
    \item $d\vert a$ and $d\vert b$.
    \item If $c\vert a$ and $c\vert b$, then $c\vert d$.
  \end{enumerate}
\end{defi}

\begin{note}
  Suppose $d\vert d'$ and $d'\vert d$. Then $d'=qd$ and $d=q'd'$. This becomes
  $d=q'qd$ or $(1-qq')d=0$. Since this is an integral domain, either $d=0$ or
  $qq'=1$, meaning $q$ and $q'$ are units. Thus, GCDs are unique only up to
  units.
\end{note}

\begin{prop}
  If $0\ne a,b\in R$ and $(a,b)=(d)$, then $d=\gcd(a,b)$.
\end{prop}

\begin{defi}
  An integral domain such that every ideal generated by two elements is
  principal is called a \define{Bezout domain}.
\end{defi}

\begin{prop}
  Let $R$ be an integral domain. If $(d)=(d')$ then there exists a unit $u\in
  R$ such that $d'=ud$.
\end{prop}

\begin{prop}
  Let $R$ be a Euclidean domain and $0\ne a,b\in R$. Let $d=r_n$ be the last
  nonzero remainder in the Euclidean algorithm. Then $d=\gcd(a,b)$ and
  $(d)=(a,b)$.
\end{prop}

\begin{prop}
  If $(d)=(a,b)$, then there exists $x,y\in R$ such that $d=ax+by$.
\end{prop}

\begin{prop}
  Consider $R=\mathbb{Z}$. If $ax+by=c$, then $c\in(d)$ so $c$ is a multiple of
  $\gcd(a,b)$.
\end{prop}

\begin{prop}
  Every ideal in a Euclidean domain is principal.
\end{prop}

\begin{pf}
  Let $I\subseteq R$ be an ideal and $R$ a Euclidean domain with norm $N$. If
  $I=\{0\}$, then $I=(0)$ is principal. Otherwise, consider $\{N(a):a\ne0,a\in
  I\}\subseteq\mathbb{N}\cup\{0\}$ as a subset of nonnegative integers. This
  set has a least element. Let $d\ne0$, $d\in I$ be an element of minimal norm.
  We show $(d)=I$. First, $d\in I$ implies that $rd\in I$ for all $r\in R$ so
  $(d)\subseteq I$. We show $I\subseteq(d)$. Let $a\in I$. Since $R$ is a
  Euclidean domain, there exists $q$ and $r$ such that $a=qd+r$, with $r=0$ or
  $N(r)<N(d)$. But $d$ has minimal norm so $r=0$. Thus, $a=qd$ and $a\in(d)$.
  Therefore, $I=(d)$ and $I$ is principal.
\end{pf}

\begin{defi}
  A \define{principal ideal domain} is an integral domain such that every ideal
  is principal.
\end{defi}

\begin{prop}
  Every Euclidean domain is a PID. This containment is proper, i.e. not every
  PID is a Euclidean domain.
\end{prop}

\begin{ex}
  $\mathbb{Z}$ is a PID. Every ideal is a subring, hence a subgroup, and hence
  is cyclic.
\end{ex}

\begin{prop}
  Let $R$ be a PID, $I\subseteq R$ a nonzero ideal. If $I$ is prime, then $I$
  is maximal.
\end{prop}

\begin{pf}
  Suppose $I\subseteq J\subseteq R$ for some ideal $J$. We show $I=J$ or $J=R$.
  Since $R$ is a PID, $I$ and $J$ are principal, so there exists $a,b\in R$
  such that $I=(a)$ and $J=(b)$. First, $I\subseteq J$, i.e. $(a)\subseteq(b)$
  so $a\in(b)$ and there exists $x\in R$ such that $a=bx$. Thus, $bx\in(a)=I$.
  But $I$ is prime, hence $b\in(a)$ or $x\in(a)$. If $b\in(a)$, then
  $(b)\subseteq(a)$ and $(a)=(b)$, i.e. $I=J$. If $x\in(a)$, there exists $y\in
  R$ such that $x=ay$. Then $x=ay=bxy=xby$. Thus, $x(by-1)=0$. Since $R$ is an
  integral domain, $x=0$ or $by-1=0$. If $x=0$, then $I=0$ but $I\ne0$ by
  assumption. Thus, $1=by$ and $b$ is a unit. Thus, $(b)=R$, i.e. $J=R$.
  Therefore, $I$ is maximal.
\end{pf}

\begin{prop}
  If $R[x]$ is a PID, then $R$ is a field.
\end{prop}

\begin{pf}
  Since $R\subseteq R[x]$, $R$ is also an integral domain. Note that
  $R[x]/(x)\cong R$ so $(x)$ is prime. Thus, $(x)$ is maximal since $R[x]$ is a
  PID. Thus, $R[x]/(x)$ is a field. But $R[x]/(x)\cong R$ so $R$ is a field.
\end{pf}

\begin{thm}[Ascending Chain Condition]
  In a PID, any strictly ascending chain of ideals is finite in length, i.e.
  $I_1\subsetneq I_2\subsetneq\ldots$ must be finite.
\end{thm}

\begin{pf}
  Let $I=\cup_n\in\mathbb{N}I_n$. This is an ideal. We are in a PID, hence $I$
  is principal, i.e. there exists $b\in R$ such that $I=(b)$. Thus, $b\in
  I=\cup_n\in\mathbb{N}I_n$. So there exists $k\in\mathbb{N}$ such that $b\in
  I_k$. Thus, $I_1, I_2,\ldots,I_{k-1}\subseteq I_k$ and $I_{k+1}\subseteq
  I_k$. Thus our chain is finite.
\end{pf}

\begin{defi}
  Let $R$ be an integral domain.
  \begin{enumerate}
    \item Let $r\in R$, $r\ne0$, and $r$ be not a unit. We say $r$ is
      \define{irreducible} in $R$ if $r=ab$ implies $a$ or $b$ is a unit in
      $R$. Otherwise, $r$ is \define{reducible}.
    \item An element $0\ne p\in R$ is called \define{prime} if $(p)$ is a prime
      ideal of $R$, i.e. $p$ is not a unit and $p\vert ab$ implies $p\vert a$
      or $p\vert b$.
    \item $a$ and $b$ are \define{associate in $R$} if there exists a unit
      $u\in R$ such that $a=ub$.
  \end{enumerate}
\end{defi}

\begin{prop}
  In an integral domain, prime elements are irreducible.
\end{prop}

\begin{pf}
  Let $p\in R$ and $(p)$ be prime. Suppose $p=ab$ where $a,b\in R$. We want to
  show that $a$ or $b$ is a unit. Note that $p\in(p)$ so $ab\in(p)$ and either
  $a\in(p)$ or $b\in(p)$. Without loss of generality, let $a\in(p)$. Then there
  exists $r\in R$ such that $a=pr$. Thus, $p=ab=prb$ so $p(1-rb)=0$. Since
  $p\ne0$ by assumption, $1=rb$ since $R$ is an integral domain so $b$ is a
  unit.
\end{pf}

\begin{note}
  Irreducible elements are not necessarily prime.
\end{note}

\begin{ex}
  Consider $3\in\mathbb{Z}[\sqrt{-5}]$. Suppose we can factor 3. Let
  $3=a(1+b\sqrt{-5})(1+c\sqrt{-5})$. Expanding gives $a(1-5bc)+5abc\sqrt{-5}$
  so $abc=0$ and $a-5abc=3$. Thus $a=3$ and $bc=0$. This means that
  $1=(1+b\sqrt{-5})(1+c\sqrt{-5})=1-5bc+bc\sqrt{-5}$ and $5bc=bc\sqrt{-5}$ so 3
  is not irreducible. However, note that $3|(1+\sqrt{-5})(1-\sqrt{-5})$.
  Another way to see this is by using norms where $N(a+b\sqrt{-5})=a^2+5b^2$
  and $N(3)=9=f^2+5g^2$.
\end{ex}

\begin{prop}
  In a PID, a nonzero element is prime if and only if it is irreducible.
\end{prop}

\begin{pf}
  The forward direction is trivial so we prove the reverse direction. Let $p\in
  R$ be irreducible. We want to show that $(p)$ is prime. But in a PID, maximal
  ideals are prime so we show that $(p)$ is maximal. Suppose $(p)\subseteq
  M\subseteq R$. Since $R$ is a PID, there exists $m\in R$ such that $M=(m)$.
  This means that $p\in(m)$ and there exists $r\in R$ such that $p=mr$. But $p$
  is irreducible so $m$ or $r$ is a unit in $R$. Thus, $(m)=R$ or $(m)=(p)$ so
  $(p)$ is maximal and therefore prime.
\end{pf}

\begin{defi}
  A \define{unique factorization domain} is an integral domain $R$ such that
  for a nonzero nonunit $r$,
  \begin{enumerate}
    \item There exists irreducible elements $p_1,\ldots,p_n\in R$ such that
      $r=p_1p_2\ldots p_n$.
    \item If $r=q_1q_2\ldots q_m$ for irreducible $q_i$, then $m=n$ and there
      exists $\sigma\in S_n$ such that $p_i$ and $q_{\sigma(i)}$ are associate.
      The $p_i$ are not necessarily distinct.
  \end{enumerate}
\end{defi}

\begin{ex}
  The following are examples and non-examples of UFDs.
  \begin{enumerate}
    \item Fields are UFDs.
    \item If $R$ is a UFD, then $R[x]$ is a UFD.
    \item $\mathbb{Z}[2i]$ is not a UFD. Note that $4=2\cdot2=(2i)(-2i)$ and
      $i\notin\mathbb{Z}[2i]=\{a+b2i:a,b\in\mathbb{Z}\}$.
  \end{enumerate}
\end{ex}

\begin{prop}
  PIDs are UFDs.
\end{prop}

\begin{pf}
  Let $r\in R$ be a nonzero nonunit. If $r$ is irreducible then we are done.
  Otherwise, $r=r_1r_2$. If $r_i$ is irreducible, we are done. Otherwise,
  $r=r_{11}r_{12}\ldots$. We then have
  $(r)\subseteq(r_1)\subseteq(r_{11})\subseteq\ldots$. This is an ascending
  chain of ideals in a PID. Thus, there exists $n\in\mathbb{N}$ such that
  $I_m=I_n$ for $m\ge n$. Then $r=r_1\ldots r_{1\ldots1}$. Uniqueness can be
  proved by induction on $n$.
\end{pf}

\begin{cor}[Fundamental Theorem of Arithmetic]
  $\mathbb{Z}$ is a UFD.
\end{cor}

\begin{note}
  There is a chain of proper containment of types of rings as follows.
  \begin{align*}
    \text{field}
    &\subsetneq\text{Euclidean domain: }\mathbb{Z}\\
    &\subsetneq\text{principal ideal domain: }\mathbb{Z}[(1+\sqrt{-19})/2]\\
    &\subsetneq\text{unique factorization domain: }\mathbb{Z}[x]\\
    &\subsetneq\text{integral domain: }\mathbb{Z}[\sqrt{-5}]\\
    &\subsetneq\text{commutative ring with 1: }\mathbb{Z}_6
  \end{align*}
\end{note}

\end{document}
