\documentclass[11pt,letterpaper]{jacky}
\usepackage[margin=1in,headheight=14pt]{geometry}
\usepackage{amsfonts, amsmath, amssymb, enumerate, fancyhdr, gensymb, lastpage, mathtools}
\usepackage{tikz-cd}

\pagestyle{fancy}
\lhead{Jacky Lee}
\chead{Real Analysis Notes}
\rhead{February 3, 2018}
\lfoot{}
\cfoot{}
\rfoot{Page\ \thepage\ of\ \pageref{LastPage}}

\linespread{1.1}
\setlength{\parindent}{0pt}

\newcommand\blankpage{
    \thispagestyle{empty}
    \addtocounter{page}{-1}
    \newpage}
\renewcommand\footrulewidth{0.4pt}

\newcommand{\Zz}{\mathbb{Z}}
\newcommand{\Rr}{\mathbb{R}}
\newcommand{\Qq}{\mathbb{Q}}
\newcommand{\Cc}{\mathbb{C}}

\newcommand{\sup}[1]{\text{sup}_{#1}}
\newcommand{\inf}[1]{\text{inf}_{#1}}

\begin{document}
\begin{center}
    \vspace*{20pt}
    \LARGE{Real Analysis Notes}
\end{center}

If $\frac{a}{b},\frac{c}{d}\in\Qq$, then
$$\frac{a}{b}+\frac{c}{d}=\frac{ad+bc}{bd} \hspace{50pt}
\frac{a}{b}-\frac{c}{d}=\frac{ad-bc}{bd} \hspace{50pt}
\frac{a}{b}\times\frac{c}{d}=\frac{ac}{bd} \hspace{50pt}
\frac{a}{b}\div\frac{c}{d}=\frac{ad}{bc}$$
provided that $\frac{c}{d}\neq\frac{0}{1}$.\\

Strictly speaking, we need to show that these operations are
\define{well-defined} or that they don't depend on the choice of
representatives from the equivalence classes.

\begin{defi}
    Suppose $S$ is an ordered set, and $E\subseteq S$. If there exists
    $\beta\in S$ such that $x\leq \beta$ for every $x\in E$, we say $E$ is
    \define{bounded above} and we call $\beta$ an \define{upper bound}. The
    terms \define{bounded below} and \define{lower bound} are defined
    similarly.
\end{defi}

\begin{defi}
        Suppose $S$ is an ordered set, $E\subseteq S$, and $E$ is bounded
        above. Suppose there exists $\alpha\in S$ such that $\alpha$ is an
        upper bound for $E$ and if $\gamma<\alpha$, then $\gamma$ is not an
        upper bound for $E$, then $\alpha$ is the \define{least upper bound} of
        $E$ or the \define{supremum} of $E$, and we write $\alpha=\sup{E}$. The
        \define{greatest lower bound} and \define{infimum} ($\inf{E}$) are
        defined similarly.
\end{defi}

\begin{ex}
    Consider the set $\{r\in\Qq:r^2<2\}$, which has no supremum in $\Qq$.
\end{ex}

\end{document}
