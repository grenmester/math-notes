\documentclass[11pt,letterpaper]{jacky}
\usepackage[margin=1in,headheight=14pt]{geometry}
\usepackage{amsfonts, amsmath, amssymb, enumerate, fancyhdr, gensymb, lastpage, mathtools}
\usepackage{tikz-cd}

\pagestyle{fancy}
\lhead{Jacky Lee}
\chead{Real Analysis Notes}
\rhead{February 3, 2018}
\lfoot{}
\cfoot{}
\rfoot{Page\ \thepage\ of\ \pageref{LastPage}}

\linespread{1.1}
\setlength{\parindent}{0pt}

\newcommand\blankpage{
  \thispagestyle{empty}
  \addtocounter{page}{-1}
  \newpage}
\renewcommand\footrulewidth{0.4pt}

\newcommand{\Zz}{\mathbb{Z}}
\newcommand{\Rr}{\mathbb{R}}
\newcommand{\Qq}{\mathbb{Q}}
\newcommand{\Cc}{\mathbb{C}}

\begin{document}
\begin{center}
  \vspace*{20pt}
  \LARGE{Real Analysis Notes}
\end{center}

If $\frac{a}{b},\frac{c}{d}\in\Qq$, then
$$\frac{a}{b}+\frac{c}{d}=\frac{ad+bc}{bd} \hspace{50pt}
\frac{a}{b}-\frac{c}{d}=\frac{ad-bc}{bd} \hspace{50pt}
\frac{a}{b}\times\frac{c}{d}=\frac{ac}{bd} \hspace{50pt}
\frac{a}{b}\div\frac{c}{d}=\frac{ad}{bc}$$
provided that $\frac{c}{d}\neq\frac{0}{1}$.\\

Strictly speaking, we need to show that these operations are
\define{well-defined} or that they don't depend on the choice of
representatives from the equivalence classes.

\begin{defi}
  Suppose $S$ is an ordered set, and $E\subseteq S$. If there exists $\beta\in
  S$ such that $x\leq \beta$ for every $x\in E$, we say $E$ is \define{bounded
  above} and we call $\beta$ an \define{upper bound}. The terms \define{bounded
  below} and \define{lower bound} are defined similarly.
\end{defi}

\begin{defi}
  Suppose $S$ is an ordered set, $E\subseteq S$, and $E$ is bounded above.
  Suppose there exists $\alpha\in S$ such that $\alpha$ is an upper bound for
  $E$ and if $\gamma<\alpha$, then $\gamma$ is not an upper bound for $E$, then
  $\alpha$ is the \define{least upper bound} of $E$ or the \define{supremum} of
  $E$, and we write $\alpha=\sup{E}$. The \define{greatest lower bound} and
  \define{infimum} ($\inf{E}$) are defined similarly.
\end{defi}

\begin{ex}
  Consider the set $\{r\in\Qq:r^2<2\}$, which has no supremum in $\Qq$.
\end{ex}

\begin{defi}
  An ordered set $S$ has the \define{least-upper-bound property} if the
  following is true: if $E\subseteq S$, $E$ is not empty, and $E$ is bounded
  above, then $\sup{E}$ exists in $S$.
\end{defi}

If an ordered set has the least-upper-bound property, then it also has the
greatest-lower-bound property.

\begin{defi}
  There exists an ordered field $\mathbb{R}$ (called the \define{real numbers})
  which has the least-upper-bound property, and it contains an isomorphic copy
  of $\mathbb{Q}$.
\end{defi}

Finite ordered fields do not exist. Consider $0\leq1\leq1+1\leq\ldots$ which
can't be a finite chain.

Any two ordered fields with the least upper-bound-property are isomorphic.

\begin{thm}
  If $x,y\in\mathbb{R}$, and $x>0$, then there is a positive integer $n$ such
  that $nx>y$. This is called the \define{Archimedean property} of
  $\mathbb{R}$.
\end{thm}

\begin{pf}
  Let $A=\{nx:n\in\mathbb{Z}^+\}$ and suppose the Archimedean property is
  false. Then $y$ would be an upper bound of $A$. But then $A$ would have a
  least upper bound. Say $\alpha=\sup A$. Since $x>0$, $\alpha-x<\alpha$, and
  $\alpha-x$ is not an upper bound. Thus, $\alpha-x<mx$ for some
  $m\in\mathbb{Z}^+$. But then $\alpha<(m+1)x$, which contradicts the fact that
  $\alpha$ is an upper bound of $A$. Thus, the Archimedean property must be
  true.
\end{pf}

\begin{thm}
  If $x,y\in\mathbb{R}$ and $x<y$, then there exists $p\in\mathbb{Q}$ such that
  $x<p<y$. We say that $\mathbb{Q}$ is \define{dense} in $\mathbb{R}$.
\end{thm}

\begin{thm}
  For every positive real number $x$ and every positive integer $n$, there is
  exactly one positive real number $y$ such that $y^n=x$.
\end{thm}

\begin{pf}
  There is at most one since $0<y_1<y_2$ implies $y_1^n<y_2^n$. Let
  $E=\{t\in\mathbb{R}:t>0,t^n<x\}$. Then $E$ is nonempty since
  $t=\frac{x}{1+x}\implies0<t<1\implies t^n<t<x\implies t\in E$. We also know
  $E$ is bounded above since $t>1+x\implies t^n>t>x\implies t\notin E$ and $t$
  is an upper bound. Define $y=\sup E$. We can then show that $y^n<x$ and
  $y^n>x$ each lead to contradictions.
\end{pf}

\end{document}
