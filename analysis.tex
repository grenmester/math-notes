\documentclass[11pt,letterpaper]{jacky}
\usepackage[margin=1in,headheight=14pt]{geometry}
\usepackage{amsfonts, amsmath, amssymb, enumerate, fancyhdr, gensymb, lastpage, mathtools}
\usepackage{tikz-cd}

\pagestyle{fancy}
\lhead{Jacky Lee}
\chead{Real Analysis Notes}
\rhead{\today}
\lfoot{}
\cfoot{}
\rfoot{Page\ \thepage\ of\ \pageref{LastPage}}

\linespread{1.1}
\setlength{\parindent}{0pt}

\newcommand\blankpage{
  \thispagestyle{empty}
  \addtocounter{page}{-1}
  \newpage}
\renewcommand\footrulewidth{0.4pt}

\newcommand{\cc}{\mathbb{C}}
\newcommand{\qq}{\mathbb{Q}}
\newcommand{\rr}{\mathbb{R}}
\newcommand{\zz}{\mathbb{Z}}

\begin{document}
\begin{center}
  \vspace*{20pt}
  \LARGE{Real Analysis Notes}
\end{center}

If $\frac{a}{b},\frac{c}{d}\in\qq$, then
$$\frac{a}{b}+\frac{c}{d}=\frac{ad+bc}{bd} \hspace{50pt}
\frac{a}{b}-\frac{c}{d}=\frac{ad-bc}{bd} \hspace{50pt}
\frac{a}{b}\times\frac{c}{d}=\frac{ac}{bd} \hspace{50pt}
\frac{a}{b}\div\frac{c}{d}=\frac{ad}{bc}$$
provided that $\frac{c}{d}\neq\frac{0}{1}$.\\

Strictly speaking, we need to show that these operations are
\define{well-defined} or that they don't depend on the choice of
representatives from the equivalence classes.

\begin{defi}
  Suppose $S$ is an ordered set, and $E\subseteq S$. If there exists $\beta\in
  S$ such that $x\leq \beta$ for every $x\in E$, we say $E$ is \define{bounded
  above} and we call $\beta$ an \define{upper bound}. The terms \define{bounded
  below} and \define{lower bound} are defined similarly.
\end{defi}

\begin{defi}
  Suppose $S$ is an ordered set, $E\subseteq S$, and $E$ is bounded above.
  Suppose there exists $\alpha\in S$ such that $\alpha$ is an upper bound for
  $E$ and if $\gamma<\alpha$, then $\gamma$ is not an upper bound for $E$, then
  $\alpha$ is the \define{least upper bound} of $E$ or the \define{supremum} of
  $E$, and we write $\alpha=\sup{E}$. The \define{greatest lower bound} and
  \define{infimum} ($\inf{E}$) are defined similarly.
\end{defi}

\begin{ex}
  Consider the set $\{r\in\qq:r^2<2\}$, which has no supremum in $\qq$.
\end{ex}

\begin{defi}
  An ordered set $S$ has the \define{least-upper-bound property} if the
  following is true: if $E\subseteq S$, $E$ is not empty, and $E$ is bounded
  above, then $\sup{E}$ exists in $S$.
\end{defi}

If an ordered set has the least-upper-bound property, then it also has the
greatest-lower-bound property.

\begin{defi}
  There exists an ordered field $\rr$ (called the \define{real numbers}) which
  has the least-upper-bound property, and it contains an isomorphic copy of
  $\qq$.
\end{defi}

Finite ordered fields do not exist. Consider $0\leq1\leq1+1\leq\ldots$ which
can't be a finite chain.

\section*{Dedekind Cuts}

\begin{enumerate}
  \item Define the elements of $\rr$ as subsets of $\qq$ called \define{cuts},
    where a cut is a subset $\alpha$ of $\qq$ such that
    \begin{enumerate}
      \item $\alpha$ is a nonempty proper subset of $\qq$ ($\alpha\ne\empty$
        and $\alpha\ne\qq$).
      \item If $p\in\alpha$, $q\in\qq$, and $q<p$, then $q\in\alpha$.
      \item If $p\in\alpha$, then $p<r$ for some $r\in\alpha$ (can't be in the
        set and be an upper bound).
    \end{enumerate}
  \item Define an order on $\rr$ where $\alpha<\beta$ if and only if $\alpha$
    is a proper subset of $\beta$.
  \item Show that the ordered set $\rr$ has the least-upper-bound property. To
    do this, suppose $A$ is a nonempty subset of $\rr$ that is bounded above.
    Let $\gamma$ be the union of all $\alpha\in A$. Then show $\gamma\in\rr$
    and $\gamma=\sup A$.
  \item For $\alpha,\beta\in\rr$, define the sum $\alpha+\beta$ to be the set
    of all sums $r+s$ where $r\in\alpha$ and $s\in\beta$. Define
    $0^*=\{t\in\qq:t<0\}$ then show axioms for addition in fields hold for
    $\rr$, and that $0^*$ is the additive identity.
  \item Show that if $\alpha,\beta,\gamma\in\rr$ and $\beta<\gamma$, then
    $\alpha+\beta<\alpha+\gamma$. This is part of showing that $\rr$ is an
    ordered field.
  \item For $\alpha,\beta\in\rr$, where $\alpha>0^*$ and $\beta>0^*$, define
    the product $\alpha\beta$ to be $\{p\in\qq:q\le rs, r\in\alpha, s\in\beta,
    r>0, s>0\}$. Note that $\alpha\beta>0^*$ if $\alpha>0^*$ and $\beta>0^*$,
    which is part of showing that $\rr$ is an ordered field.
  \item Extend the definition of multiplication to all of $\rr$ by setting, for
    all $\alpha,\beta\in\rr$, $\alpha0^*=0^*\alpha=0^*$ and
    $$\alpha\beta=\begin{cases}
      (-\alpha)(-\beta) & \alpha<0^*, \beta<0^*\\
      -[(-\alpha)(\beta)] & \alpha<0^*, \beta>0^*\\
      -[(\alpha)(-\beta)] & \alpha>0^*, \beta<0^*
    \end{cases}$$
    then prove the distributive law.
  \item Associate to each $r\in\qq$ the real number $r^*=\{t\in\qq:t<r\}$ and
    let $\qq^*=\{r^*:r\in\qq\}$. These are the rational cuts in $\rr$.
  \item Show that $\qq$ is isomorphic to $\qq^*$ as ordered fields.
\end{enumerate}

Any two ordered fields with the least upper-bound-property are isomorphic.

\begin{thm}
  If $x,y\in\rr$, and $x>0$, then there is a positive integer $n$ such that
  $nx>y$. This is called the \define{Archimedean property} of $\rr$.
\end{thm}

\begin{pf}
  Let $A=\{nx:n\in\zz^+\}$ and suppose the Archimedean property is false. Then
  $y$ would be an upper bound of $A$. But then $A$ would have a least upper
  bound. Say $\alpha=\sup A$. Since $x>0$, $\alpha-x<\alpha$, and $\alpha-x$ is
  not an upper bound. Thus, $\alpha-x<mx$ for some $m\in\zz^+$. But then
  $\alpha<(m+1)x$, which contradicts the fact that $\alpha$ is an upper bound
  of $A$. Thus, the Archimedean property must be true.
\end{pf}

\begin{thm}
  If $x,y\in\rr$ and $x<y$, then there exists $p\in\qq$ such that $x<p<y$. We
  say that $\qq$ is \define{dense} in $\rr$.
\end{thm}

\begin{thm}
  For every positive real number $x$ and every positive integer $n$, there is
  exactly one positive real number $y$ such that $y^n=x$.
\end{thm}

\begin{pf}
  There is at most one since $0<y_1<y_2$ implies $y_1^n<y_2^n$. Let
  $E=\{t\in\rr:t>0,t^n<x\}$. Then $E$ is nonempty since
  $t=\frac{x}{1+x}\implies0<t<1\implies t^n<t<x\implies t\in E$. We also know
  $E$ is bounded above since $t>1+x\implies t^n>t>x\implies t\notin E$ and $t$
  is an upper bound. Define $y=\sup E$. We can then show that $y^n<x$ and
  $y^n>x$ each lead to contradictions.
\end{pf}

\end{document}
